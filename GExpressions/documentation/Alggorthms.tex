% ----------------------------------------------------------------
% AMS-LaTeX Paper ************************************************
% **** -----------------------------------------------------------
% documentclass[fleqn]{report}
\documentclass[a4paper,fleqn]{article}
% \documentclass[a4paper,fleqn]{scrreprt}
\setlength{\oddsidemargin}{0.0cm}
\setlength{\evensidemargin}{0.0cm}
\setlength{\textwidth}{15.0cm}
\setlength{\paperwidth}{210mm}
\setlength{\paperheight}{297mm}
\usepackage{graphicx} 
\usepackage[utf8]{inputenc}
\usepackage{amsmath}
\usepackage{amssymb}
\usepackage{alltt}
\usepackage{color}
\usepackage{fancyvrb}
%\usepackage{hyperref}

\newcounter{thm}[section]
\newtheorem{theorem}{Theorem}[section]
\newtheorem{lemma}[theorem]{Lemma}
\newtheorem{proposition}[theorem]{Hilfssatz}
\newtheorem{corollary}[theorem]{Corollary}
\newtheorem{definition}[theorem]{Definition}

\newenvironment{proof}[1][Beweis]{\begin{trivlist}
\item[\hskip \labelsep {\bfseries #1}]}{\end{trivlist}}
\newcommand*\codeRed[1]{\textcolor[rgb]{1,0,0}{\textbf{#1}}}
\newcommand*\codeGreen[1]{\textcolor[rgb]{0,0.5,0}{\textbf{#1}}}
\newcommand{\lb}{[}
\newcommand{\rb}{]}
\newcommand{\codeCall}[1]{\textbf{\texttt{#1}}}
\newcommand{\mcL}{\mathcal{L}}
%\newenvironment{\vspace{1ex}\noindent{\bf Proof}\hspace{0.5em}}
%	{\hfill\qed\vspace{1ex}}
 

\newenvironment{example}[1][Example]{\begin{trivlist}
\item[\hskip \labelsep {\bfseries #1}]}{\end{trivlist}}
\newenvironment{remark}[1][Remark]{\begin{trivlist}
\item[\hskip \labelsep {\bfseries #1}]}{\end{trivlist}}

\newcommand{\qed}{\nobreak \ifvmode \relax \else
      \ifdim\lastskip<1.5em \hskip-\lastskip
      \hskip1.5em plus0em minus0.5em \fi \nobreak
      \vrule height0.75em width0.5em depth0.25em\fi}

\newcommand{\qt}{\texttt}

\newcommand{\bibdef}[4]{\bibitem{#1} {\bf #2:} {\it #3}\\#4}

\newcommand{\eeeccc}{\end{section}}
\newcommand{\gauss}[1]{\lfloor #1\rfloor}
\numberwithin{equation}{section}

\newcommand{\beginchap}[1]{\begin{section}{#1}
%\setcounter{equation}{1}
}
\setcounter{tocdepth}{4}
\setcounter{secnumdepth}{4}


\title{Data Structures and Algorithms in the GExpressions Package}
\author{Hugo Meder}
\begin{document}
\begin{titlepage}
\maketitle
\end{titlepage}
\tableofcontents

\beginchap{Data structures}
\begin{subsection}{Terms}
Terms are represented by GTerm class instances. They consist of the elements:
\begin{itemize}
	\item a factor (\texttt{double}).
	\item a list (\texttt{Vector}) of inverse metric tensors $g^{\mu\nu}$ (\texttt{GUpper}).
	\item a list (\texttt{Vector}) of lower external indices $_\sigma$ (\texttt{LowerExtern}).
	\item a list (\texttt{Vector}) of metric tensors and its derivatives $g_{\mu\nu,\sigma\tau...}$ (\texttt{Integer}).
\end{itemize}

\begin{subsubsection}{The Factor}
Its meaning is just to multiply the term by this factor.
\end{subsubsection}

\begin{subsubsection}{The List of metric Tensors and its Derivatives}
This list consists only of \texttt{Inerger}s. I does not contain any information of the index structure of its elements.
This is supplied by the other elements of the \texttt{GTerm} instances.
\end{subsubsection}

\begin{subsubsection}{The \texttt{Index} class}
The \texttt{Index} class instances provide the following information:
\begin{itemize}
	\item if it is an external index, then the index is simply representen by an \texttt{Integer}.
	\item if it is not an external index, that it is an internal index, then it is identified, first by an \texttt{Integer} referring to the list of 
	metric tensors and its derivatives, and second by a \texttt{boolean} which decides whether this index refers to the metric tensor itself or its derivative.
\end{itemize}
\end{subsubsection}

\begin{subsubsection}{The List of inverse metric Tensors}
The elements of that list (\texttt{GUpper}) consist just of the two upper indices (\texttt{Index}).
\end{subsubsection}

\begin{subsubsection}{The List of lower external Indices}
The elements of that list (\texttt{LowerExtern}) consist of an \texttt{Index} element which is of the enternal type, and one of the internal type.
\end{subsubsection}

\begin{subsubsection}{The Construction of \texttt{GTerm} Objects}
First a \texttt{GTerm} object is created providing the factor and the list of metric tensors and its derivatives as an \texttt{integer[]} array by its constructor. For each of the last array an internal object is created. Then elements of \texttt{GUpper} and \texttt{LowerExtern} are added. The interal objects count the number of internal indices. Finally a method \texttt{finish()} is called, which ensures that the numbers of internal indices match for each of the internal objects.
\end{subsubsection}

\end{subsection}
\eeeccc
\end{document}
