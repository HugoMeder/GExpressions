% ----------------------------------------------------------------
% AMS-LaTeX Paper ************************************************
% **** -----------------------------------------------------------
% documentclass[fleqn]{report}
\documentclass[a4paper,fleqn]{article}
% \documentclass[a4paper,fleqn]{scrreprt}
\setlength{\oddsidemargin}{0.0cm}
\setlength{\evensidemargin}{0.0cm}
\setlength{\textwidth}{15.0cm}
\setlength{\paperwidth}{210mm}
\setlength{\paperheight}{297mm}
\usepackage{german}
\usepackage{graphicx} 
\usepackage[utf8]{inputenc}
\usepackage{amsmath}
\usepackage{amssymb}
\usepackage{alltt}
\usepackage{color}
\usepackage{fancyvrb}
%\usepackage{hyperref}

\newcounter{thm}[section]
\newtheorem{theorem}{Theorem}[section]
\newtheorem{lemma}[theorem]{Lemma}
\newtheorem{proposition}[theorem]{Hilfssatz}
\newtheorem{corollary}[theorem]{Corollary}
\newtheorem{definition}[theorem]{Definition}

\newenvironment{proof}[1][Beweis]{\begin{trivlist}
\item[\hskip \labelsep {\bfseries #1}]}{\end{trivlist}}
\newcommand*\codeRed[1]{\textcolor[rgb]{1,0,0}{\textbf{#1}}}
\newcommand*\codeGreen[1]{\textcolor[rgb]{0,0.5,0}{\textbf{#1}}}
\newcommand{\lb}{[}
\newcommand{\rb}{]}
\newcommand{\codeCall}[1]{\textbf{\texttt{#1}}}
\newcommand{\mcL}{\mathcal{L}}
%\newenvironment{\vspace{1ex}\noindent{\bf Proof}\hspace{0.5em}}
%	{\hfill\qed\vspace{1ex}}
 

\newenvironment{example}[1][Example]{\begin{trivlist}
\item[\hskip \labelsep {\bfseries #1}]}{\end{trivlist}}
\newenvironment{remark}[1][Remark]{\begin{trivlist}
\item[\hskip \labelsep {\bfseries #1}]}{\end{trivlist}}

\newcommand{\qed}{\nobreak \ifvmode \relax \else
      \ifdim\lastskip<1.5em \hskip-\lastskip
      \hskip1.5em plus0em minus0.5em \fi \nobreak
      \vrule height0.75em width0.5em depth0.25em\fi}

\newcommand{\qt}{\texttt}

\newcommand{\bibdef}[4]{\bibitem{#1} {\bf #2:} {\it #3}\\#4}

\newcommand{\eeeccc}{\end{section}}
\newcommand{\gauss}[1]{\lfloor #1\rfloor}
\numberwithin{equation}{section}

\newcommand{\beginchap}[1]{\begin{section}{#1}
%\setcounter{equation}{1}
}
\setcounter{tocdepth}{4}
\setcounter{secnumdepth}{4}


\title{Gravitations-Energie in der allgemeine Relativitätstheorie}
\author{Hugo Meder}
\begin{document}
\begin{titlepage}
\maketitle
\end{titlepage}
\tableofcontents
\beginchap{Die Methodik}
\begin{subsection}{Das Problem}
Aus einer Wirkung
\begin{eqnarray}
\label{lagrange_total}
S &=& \int dx^0dx^1dx^2dx^3\ \sqrt{-g}\left(\mathcal{L}-\frac{1}{2\kappa}R\right)
\end{eqnarray}
sollen durch Variation sowohl die Einsteinschen Feldgleichungen als auch die Bewegungsgleichungen hervorgehen. Aus einem allgemeinen Zusammenhang zwischen Variation und dem Noetherschen Theorem können dann Ausdrücke
für Energie und Impuls hergeleitet werden. Diese Ausdrücke, so die Erwartung, können als Summe von gravitativer und materieller Energie geschrieben werden.
Konkret werden wir zeigen, das die Variation von $S$ (zumindest für zwei beispielhafte Felder) die Form
\begin{eqnarray}
\label{total_variation_integral}
\delta S &=& \int dx^0dx^1dx^2dx^3\ \sqrt{-g}\left(\frac{1}{2}\left(-T^{\mu\nu}+\frac{1}{\kappa}G^{\mu\nu}\right)\delta g_{\mu\nu}+F\delta\phi\right)
\end{eqnarray}
annimmt, woraus sich
\begin{eqnarray*}
G^{\mu\nu} &=& \kappa T^{\mu\nu}\\
F &=& 0
\end{eqnarray*}
ergibt. Hier ist $G^{\mu\nu}$ der Eintein-Tensor, $T^{\mu\nu}$ ein symmetrischer und divergenzfreier Energie-Impuls-Tensor der Materie, und $F$ die Bewegungsgleichung für Materie-Felder.
\end{subsection}

\begin{subsection}{Der allgemeine Zusammenhang zwischen Variationsprinzip und Erhaltungsgrößen}
Zum Studium des allgemeinen Zusammenhangs benutzen wir folgende Wirkung
\begin{eqnarray*}
S &=& \int dx^0dx^1dx^2dx^3\ \mathcal{L}(\phi, \phi_{,\mu}, \phi_{,\mu\nu})
\end{eqnarray*}
Dadurch, dass wir $\mathcal{L}$ auch von den zweiten abhängen lassen, tritt zum einen das allgemeine Muster besser zum Vorschein, zum anderen hängt der Ricci-Skalar $R$ tatsächlich von zweiten Ableitungen $g_{\mu\nu,\alpha\beta}$ ab.
\begin{subsubsection}{Die Bewegungsgleichungen}
Wir führen nun die Variation explizit durch:
\begin{eqnarray*}
\delta S &=& \int dx^0dx^1dx^2dx^3\ \delta \mathcal{L}=\int dx^0dx^1dx^2dx^3\ \frac{\partial \mathcal{L}}{\partial \phi}\delta \phi+\frac{\partial \mathcal{L}}{\partial \phi_{,\mu}}\delta \phi_{,\mu}+\frac{\partial \mathcal{L}}{\partial \phi_{,\mu\nu}}\delta \phi_{,\mu\nu}
\end{eqnarray*}
und benutzen partielle Integration:
\begin{eqnarray*}
\int dx^0dx^1dx^2dx^3\ \frac{\partial \mathcal{L}}{\partial \phi_{,\mu}}\delta \phi_{,\mu}&=& \int dx^0dx^1dx^2dx^3\ \left(\frac{\partial \mathcal{L}}{\partial \phi_{,\mu}}\delta \phi\right)_{,\mu}-\left(\frac{\partial \mathcal{L}}{\partial \phi_{,\mu}}\right)_{,\mu}\delta \phi\\
\int dx^0dx^1dx^2dx^3\ \frac{\partial \mathcal{L}}{\partial \phi_{,\mu\nu}}\delta \phi_{,\mu\nu}&=& \int dx^0dx^1dx^2dx^3\ \left(\frac{\partial \mathcal{L}}{\partial \phi_{,\mu\nu}}\delta \phi_{,\mu}\right)_{,\nu}-\left(\frac{\partial \mathcal{L}}{\partial \phi_{,\mu\nu}}\right)_{,\nu}\delta \phi_{,\mu}\\
&=& \int dx^0dx^1dx^2dx^3\ \left[\left(\frac{\partial \mathcal{L}}{\partial \phi_{,\mu\nu}}\delta \phi_{,\mu}\right)_{,\nu}\right.\\
&&\left.-\left(\left(\frac{\partial \mathcal{L}}{\partial \phi_{,\mu\nu}}\right)_{,\nu}\delta \phi\right)_{,\mu}+\left(\frac{\partial \mathcal{L}}{\partial \phi_{,\mu\nu}}\right)_{,\nu\mu}\delta \phi\right]\\
\end{eqnarray*}
und erhalten
\begin{eqnarray*}
\delta S &=& \int dx^0dx^1dx^2dx^3\ \left(\frac{\partial \mathcal{L}}{\partial \phi}-\left(\frac{\partial \mathcal{L}}{\partial \phi_{,\mu}}\right)_{,\mu}+\left(\frac{\partial \mathcal{L}}{\partial \phi_{,\mu\nu}}\right)_{,\nu\mu}\right)\delta \phi
\end{eqnarray*}
wobei wir die Integration über die Divergenzen fallen gelassen haben\footnote{Dies ist aber nur legitim, falls wegen $\left(\frac{\partial \mathcal{L}}{\partial \phi_{,\mu\nu}}\delta \phi_{,\mu}\right)_{,\nu}$ am Rand nebst den Variationen $\delta\phi$ auch deren Ableitungen $\delta\phi_{,\mu}$ verschwinden.}. Damit erhalten wir die Euler-Lagrangen Bewegungsgleichungen:
\begin{eqnarray}
\label{eul_lagr_eqns}
\frac{\partial \mathcal{L}}{\partial \phi}-\left(\frac{\partial \mathcal{L}}{\partial \phi_{,\mu}}\right)_{,\mu}+\left(\frac{\partial \mathcal{L}}{\partial \phi_{,\mu\nu}}\right)_{,\nu\mu} &=& 0
\end{eqnarray}
Sollte $\mathcal{L}$ nur von den ersten Ableitungen $\phi_{,\mu}$ abhängen reduzieren sich die Bewegungsgleichungen zu
\begin{eqnarray*}
\frac{\partial \mathcal{L}}{\partial \phi}-\left(\frac{\partial \mathcal{L}}{\partial \phi_{,\mu}}\right)_{,\mu} &=& 0
\end{eqnarray*}
Das Wegfallen der Divergenzen ist dadurch gerechtfertigt, das sich die Variation über ein \textbf{Integral} erstreckt und angenommen wird dass die Randterme verschwinden (partielle Integration). Dennoch sei die Variation $\delta \mathcal{L}$ auch zur späteren Verwendung, vollständig hingeschrieben:
\begin{eqnarray}
\label{l_tot_vari}
\delta \mathcal{L} &=& \left(\frac{\partial \mathcal{L}}{\partial \phi}-\left(\frac{\partial \mathcal{L}}{\partial \phi_{,\mu}}\right)_{,\mu}+\left(\frac{\partial \mathcal{L}}{\partial \phi_{,\mu\nu}}\right)_{,\nu\mu}\right)\delta \phi\cr
&&+\left(\frac{\partial \mathcal{L}}{\partial \phi_{,\mu}}\delta \phi\right)_{,\mu}+\left(\frac{\partial \mathcal{L}}{\partial \phi_{,\mu\nu}}\delta \phi_{,\mu}\right)_{,\nu}
-\left(\left(\frac{\partial \mathcal{L}}{\partial \phi_{,\mu\nu}}\right)_{,\nu}\delta \phi\right)_{,\mu}\cr
&=& \left(\frac{\partial \mathcal{L}}{\partial \phi}-\left(\frac{\partial \mathcal{L}}{\partial \phi_{,\mu}}\right)_{,\mu}+\left(\frac{\partial \mathcal{L}}{\partial \phi_{,\mu\nu}}\right)_{,\nu\mu}\right)\delta \phi\cr
&&+\left(\frac{\partial \mathcal{L}}{\partial \phi_{,\mu}}\delta \phi+\frac{\partial \mathcal{L}}{\partial \phi_{,\mu\nu}}\delta \phi_{,\nu}
-\left(\frac{\partial \mathcal{L}}{\partial \phi_{,\mu\nu}}\right)_{,\nu}\delta \phi\right)_{,\mu}
\end{eqnarray}
\end{subsubsection}

\begin{subsubsection}{Noether's Theorem und Erhaltungsgrößen}
Noether's Theorem beschreibt die Existenz einer Erhaltungsgröße falls die Lagrange-Dichte $\mathcal{L}$ eine kontinuierliche Symmetrie-Bedingung
\begin{eqnarray}
\label{noether_cond}
\frac{d\mathcal{L}}{d\epsilon} &=& D^\mu_{,\mu}
\end{eqnarray}
erfüllt. Wir können \ref{l_tot_vari} verwenden:
\begin{eqnarray*}
\frac{d\mathcal{L}}{d\epsilon} &=& \left(\frac{\partial \mathcal{L}}{\partial \phi}-\left(\frac{\partial \mathcal{L}}{\partial \phi_{,\mu}}\right)_{,\mu}+\left(\frac{\partial \mathcal{L}}{\partial \phi_{,\mu\nu}}\right)_{,\nu\mu}\right)\frac{d\phi}{d\epsilon}\cr
&&+\left(\frac{\partial \mathcal{L}}{\partial \phi_{,\mu}}\frac{d\phi}{d\epsilon}+\frac{\partial \mathcal{L}}{\partial \phi_{,\mu\nu}}\frac{d\phi_{,\mu}}{d\epsilon}
-\left(\frac{\partial \mathcal{L}}{\partial \phi_{,\mu\nu}}\right)_{,\nu}\frac{d\phi}{d\epsilon}\right)_{,\mu}= D^\mu_{,\mu}
\end{eqnarray*}
Nehmen wir nun an, die Bewegungsgleichungen \ref{eul_lagr_eqns} seien erfüllt, dann wird daraus
\begin{eqnarray*}
\frac{d\mathcal{L}}{d\epsilon} &=&\left(\frac{\partial \mathcal{L}}{\partial \phi_{,\mu}}\frac{d\phi}{d\epsilon}+\frac{\partial \mathcal{L}}{\partial \phi_{,\mu\nu}}\frac{d\phi_{,\mu}}{d\epsilon}
-\left(\frac{\partial \mathcal{L}}{\partial \phi_{,\mu\nu}}\right)_{,\nu}\frac{d\phi}{d\epsilon}\right)_{,\mu}= D^\mu_{,\mu}
\end{eqnarray*}
und wir erhalten einen divergenzfreien Strom $j^\mu$
\begin{eqnarray*}
j^\mu_{,\mu} &=& \left(\frac{\partial \mathcal{L}}{\partial \phi_{,\mu}}\frac{d\phi}{d\epsilon}+\frac{\partial \mathcal{L}}{\partial \phi_{,\mu\nu}}\frac{d\phi_{,\mu}}{d\epsilon}
-\left(\frac{\partial \mathcal{L}}{\partial \phi_{,\mu\nu}}\right)_{,\nu}\frac{d\phi}{d\epsilon}-D^\mu\right)_{,\mu}=0
\end{eqnarray*}
Die entsprechende globale Erhaltungsgröße ist
\begin{eqnarray*}
Q (x^0) &=& \int dx^1dx^2dx^3\ j^0(x^\mu)
\end{eqnarray*}
denn
\begin{eqnarray*}
\frac{d}{dt}Q (x^0) &=& cQ_{,0}  = c\int dx^1dx^2dx^3\ j^0_{,0}(x^\mu) = c\int dx^1dx^2dx^3\ j^\mu_{,\mu}-j^k_{,k}\\
&=& -c\int dx^1dx^2dx^3\ j^k_{,k} = 0
\end{eqnarray*}
\end{subsubsection}

\begin{subsubsection}{Energie und Impuls}
Für eine infinitesimale Verschiebung
\begin{eqnarray*}
\phi(x^\mu) &\rightarrow& \phi(x^\mu+\xi^\mu\epsilon)
\end{eqnarray*}
mit konstantem $\xi^\mu$ erhalten wir
\begin{eqnarray*}
\frac{d \phi}{d\epsilon} &=& \phi_{,\mu}\xi^\mu\\
\frac{d \mathcal{L}}{d\epsilon} &=& \mathcal{L}_{,\mu}\xi^\mu = \left(\mathcal{L}\xi^\mu\right)_{,\mu}
\end{eqnarray*}
womit die Bedingung \ref{noether_cond} mit $D^\mu = \mathcal{L}\xi^\mu$ erfüllt ist. Der entsprechende divergenzfreie Strom ist dann
\begin{eqnarray*}
j^\mu &=& \left(\frac{\partial \mathcal{L}}{\partial \phi_{,\mu}}\frac{d\phi}{d\epsilon}+\frac{\partial \mathcal{L}}{\partial \phi_{,\mu\nu}}\frac{d\phi_{,\mu}}{d\epsilon}
-\left(\frac{\partial \mathcal{L}}{\partial \phi_{,\mu\nu}}\right)_{,\nu}\frac{d\phi}{d\epsilon}-D^\mu\right)\\
&=& \left(\frac{\partial \mathcal{L}}{\partial \phi_{,\mu}}\phi_{,\lambda}+\frac{\partial \mathcal{L}}{\partial \phi_{,\mu\nu}}\phi_{,\mu\lambda}
-\left(\frac{\partial \mathcal{L}}{\partial \phi_{,\mu\nu}}\right)_{,\nu}\phi_{,\lambda}-\mathcal{L}\delta^\mu_\lambda\right)\xi^\lambda
\end{eqnarray*}
mit
\begin{eqnarray*}
\Theta^\mu_\lambda &=& \left(\frac{\partial \mathcal{L}}{\partial \phi_{,\mu}}\phi_{,\lambda}+\frac{\partial \mathcal{L}}{\partial \phi_{,\mu\nu}}\phi_{,\mu\lambda}
-\left(\frac{\partial \mathcal{L}}{\partial \phi_{,\mu\nu}}\right)_{,\nu}\phi_{,\lambda}-\mathcal{L}\delta^\mu_\lambda\right)
\end{eqnarray*}
erhalten wir den globalen Erhaltungssatz
\begin{eqnarray*}
Q &=& \int dx^1dx^2dx^3\ \Theta^0_\lambda\xi^\lambda
\end{eqnarray*}
Für $\xi^\mu=\delta^\mu_0$ beschreibt $Q$ die Energie, für Raumindizes $k$ und $\xi^\mu=\delta^\mu_k$ bedeutet Q den Impuls in $k$-Richtung. Der Energie-Impuls Vektor ist entsprechend
\begin{eqnarray*}
P_\lambda &=& \int dx^1dx^2dx^3\ \Theta^0_\lambda
\end{eqnarray*}
Die Größe $\Theta^\mu_\lambda$ heißt \textbf{kanonischer} Energie-Impuls-Tensor.
\end{subsubsection}
\end{subsection}
\eeeccc


\beginchap{Lagrange Formulierung der Vakuumgleichungen der allgemeinen Relativitätstheorie}
\label{art_lagr}

\begin{subsection}{Integration in gekrümmten Räumen}
Integrale über Vektoren und Tensoren in gekrümmten Räumen sind naturgemäß sinnlos, da Vektoren und Tensoren an verschiedene Punkten im Raume zu verschiedenen Tangential-Räumen gehören und daher nicht addiert werden können.
Dieses Problem existiert nicht für Skalare. Integrale der Form
\begin{eqnarray*}
\int_\Omega dx^0dx^1dx^2dx^3 \sqrt{-g}f
\end{eqnarray*}
sind zum Beispiel für ein vierdimensionales Raum-Zeit-Gebiet $\Omega$ sehr wohl definiert.

\begin{subsubsection}{Der Gaußsche Satz und partielle Integration in gekrümmten Räumen}
\begin{paragraph}{Kovariante Divergenz}
Die Anwendung des Satzes von Gauß im Rahmen der Riemannschen Geometrie beruht im Wesentlichen auf der folgenden Identität:
\begin{eqnarray}
\label{riemann_gauss}
\sqrt{g}F^\mu_{;\mu} &=& \left(\sqrt{g}F^\mu\right)_{,\mu}
\end{eqnarray}
\begin{proof}
Zunächst stellen wir fest:
\begin{eqnarray*}
F^\mu_{;\nu} &=& F^\mu_{,\nu}+F^\alpha\Gamma^\mu_{\alpha\nu}\\
F^\mu_{;\mu} &=& F^\mu_{,\mu}+F^\alpha\Gamma^\mu_{\alpha\mu}=F^\mu_{,\mu}+F^\alpha\frac{1}{2}g^{\mu\nu}g_{\mu\nu,\alpha}\\
\end{eqnarray*}
und weiter gilt es den Ausdruck $\sqrt{g}_{,\mu}$ weiter auszuwerten:
\begin{eqnarray*}
\sqrt{g}_{,\mu} &=& \frac{1}{2}\frac{g_{,\mu}}{\sqrt{g}}\\
\det(g_{\alpha\beta}(x^\mu+d\gamma^\mu)) &=& \det(g_{\alpha\beta}(x)+g_{\alpha\beta,\mu}(x)d\gamma^\mu)\\
&=& \det\left(g_{\alpha\sigma}\left(\delta^\sigma_\beta+g^{\sigma\gamma}g_{\gamma\beta,\mu}d\gamma^\mu\right)\right)\\
&=& \det\left(g_{\alpha\sigma}\right)\det\left(\delta^\sigma_\beta+g^{\sigma\gamma}g_{\gamma\beta,\mu}d\gamma^\mu\right)\\
&=& g\left(1+g^{\beta\gamma}g_{\gamma\beta,\mu}d\gamma^\mu\right)\\
g_{,\mu} &=&gg^{\beta\gamma}g_{\gamma\beta,\mu}\\
\sqrt{g}_{,\mu} &=& \frac{1}{2}\frac{gg^{\beta\gamma}g_{\gamma\beta,\mu}}{\sqrt{g}}=\frac{1}{2}\sqrt{g}g^{\beta\gamma}g_{\gamma\beta,\mu}\\
\end{eqnarray*}
Damit erhalten wir schließlich
\begin{eqnarray*}
\sqrt{g}F^\mu_{;\mu} &=& \sqrt{g}\left(F^\mu_{,\mu}+F^\alpha\frac{1}{2}g^{\mu\nu}g_{\mu\nu,\alpha}\right)\\
\left(\sqrt{g}F^\mu\right)_{,\mu} &=& \sqrt{g}F^\mu_{,\mu}+\sqrt{g}_{,\mu}F^\mu=\sqrt{g}\left(F^\mu_{,\mu}+\frac{1}{2}F^\mu g^{\beta\gamma}g_{\gamma\beta,\mu}\right)
\end{eqnarray*}
\qed
\end{proof}
\end{paragraph}

\begin{paragraph}{Satz von Gauß}
Der Satz von Gauß lässt sich nun für eine vierdimensionale Raum-Zeit formulieren:
\begin{eqnarray*}
\int_\Omega dx^0dx^1dx^2dx^3 \sqrt{-g}F^\mu_{;\mu}&=&\int_\Omega dx^0dx^1dx^2dx^3 \left(\sqrt{-g}F^\mu\right)_{,\mu}\\
&=&\oint_{\partial \Omega} d \omega_\mu \sqrt{-g}F^\mu
\end{eqnarray*}
für ein beliebiges Raum-Zeit-Gebiet $\Omega$. Erstreckt sich $\Omega$ aber über den gesamten Raum, aber über ein beschränktes Zeitintervall $x^0\in [a,b]$ dann ergibt sich folgende Form:
\begin{eqnarray*}
\int_{x^0=a}^{x^0=b} dx^0dx^1dx^2dx^3 \sqrt{-g}F^\mu_{;\mu}&=&\left(\int dx^1dx^2dx^3 \sqrt{-g}F^0\right|_{x^0=b}-\left(\int dx^1dx^2dx^3 \sqrt{-g}F^0\right|_{x^0=a}
\end{eqnarray*}
\end{paragraph}
\end{subsubsection}
\end{subsection}

\begin{subsection}{Die These}
Wir betrachten die so genannte Einstein-Hilbert-Wirkung (bis auf eine Konstante)
\begin{eqnarray*}
S[g] &=& \int \sqrt{-g} Rdx^4
\end{eqnarray*}
und wir behaupten, dass
\begin{eqnarray*}
\delta S[g] = 0 &\Rightarrow& G_{\mu\nu} = 0
\end{eqnarray*}
genauer:
\begin{eqnarray}
\label{einst_hilbert}
\delta S[g] &=& c \int dx^4\ \sqrt{-g(x)}G^{\mu\nu}(x)\delta g_{\mu\nu} = 0
\end{eqnarray}
mit einer Konstanten $c\ne 0$.
\end{subsection}
\begin{subsection}{Hilfssatz: $\delta\Gamma^\sigma_{\mu\nu}$ ist ein Tensor}
Einfachheitshalber untersuchen wir zuerst
\begin{eqnarray*}
\Gamma_{\mu\nu\lambda} &=& g_{\lambda\alpha}\Gamma^\alpha_{\mu\nu}\\
&=& \frac{1}{2}\left(g_{\lambda\nu,\mu}+g_{\mu\lambda,\nu}-g_{\mu\nu,\lambda}\right)\\
\delta\Gamma_{\mu\nu\lambda} &=& \frac{1}{2}\left(\delta g_{\lambda\nu,\mu}+\delta g_{\mu\lambda,\nu}-\delta g_{\mu\nu,\lambda}\right)
\end{eqnarray*}
Anstelle der Ableitungen können auch die kovarianten Ableitungen benutzt werden:
\begin{eqnarray*}
\delta g_{\lambda\nu;\mu} &=& \delta g_{\lambda\nu,\mu}-\delta g_{\alpha\nu}\Gamma^\alpha_{\lambda\mu}-\delta g_{\lambda\alpha}\Gamma^\alpha_{\nu\mu}\\
\delta g_{\lambda\mu;\nu} &=& \delta g_{\lambda\mu,\nu}-\delta g_{\alpha\mu}\Gamma^\alpha_{\lambda\nu}-\delta g_{\lambda\alpha}\Gamma^\alpha_{\mu\nu}\\
-\delta g_{\mu\nu;\lambda} &=& -\delta g_{\mu\nu,\lambda}+\delta g_{\alpha\nu}\Gamma^\alpha_{\mu\lambda}+\delta g_{\mu\alpha}\Gamma^\alpha_{\nu\lambda}\\
\frac{1}{2}\left(\delta g_{\lambda\nu;\mu}+\delta g_{\mu\lambda;\nu}-\delta g_{\mu\nu;\lambda}\right) &=& \delta\Gamma_{\mu\nu\lambda}-\delta g_{\lambda\alpha}\Gamma^\alpha_{\mu\nu}\\
\delta\Gamma_{\mu\nu\lambda} &=& \frac{1}{2}\left(\delta g_{\lambda\nu;\mu}+\delta g_{\mu\lambda;\nu}-\delta g_{\mu\nu;\lambda}\right)+\delta g_{\lambda\alpha}\Gamma^\alpha_{\mu\nu}
\end{eqnarray*}
Und schließlich die Variation von $\delta\Gamma^\sigma_{\mu\nu}$:
\begin{eqnarray*}
\Gamma^\sigma_{\mu\nu} &=& g^{\sigma\lambda}\Gamma_{\mu\nu\lambda}\\
\delta\Gamma^\sigma_{\mu\nu} &=& \delta g^{\sigma\lambda}\Gamma_{\mu\nu\lambda}+g^{\sigma\lambda}\delta\Gamma_{\mu\nu\lambda}\\
&=& -\delta g_{\alpha\beta}g^{\alpha\sigma}g^{\beta\lambda}\Gamma_{\mu\nu\lambda}+g^{\sigma\lambda}\delta\Gamma_{\mu\nu\lambda}\\
&=& -\delta g_{\alpha\beta}g^{\alpha\sigma}\Gamma^\beta_{\mu\nu}+g^{\sigma\lambda}\delta\Gamma_{\mu\nu\lambda}\\
&=& -\delta g_{\alpha\beta}g^{\alpha\sigma}\Gamma^\beta_{\mu\nu}+g^{\sigma\lambda}\left(\frac{1}{2}\left(\delta g_{\lambda\nu;\mu}+\delta g_{\mu\lambda;\nu}-\delta g_{\mu\nu;\lambda}\right)+\delta g_{\lambda\alpha}\Gamma^\alpha_{\mu\nu}\right)\\
&=& \frac{1}{2}g^{\sigma\lambda}\left(\delta g_{\lambda\nu;\mu}+\delta g_{\mu\lambda;\nu}-\delta g_{\mu\nu;\lambda}\right)\\
\end{eqnarray*}
Die letzte Zeile beschreibt einen Tensor, denn $g_{\mu\nu}$ und $\delta g_{\mu\nu}$ sind Tensoren und die kovariante Ableitung macht aus Tensoren wiederum Tensoren.
\end{subsection}

\begin{subsection}{Die Variation}
Wir wollen die Variation
\begin{eqnarray*}
\delta\int\sqrt{-g}Rd^4x
\end{eqnarray*}
bestimmen.
Der Riemannsche Krümmungstensor ist  gegeben durch:
\begin{eqnarray*}
R^\alpha_{\mu\nu\lambda} &=& -\Gamma^\alpha_{\mu\nu,\lambda}
+\Gamma^\alpha_{\mu\lambda,\nu}+\Gamma^\alpha_{\beta\nu}\Gamma^\beta_{\mu\lambda}-\Gamma^\alpha_{\beta\lambda}\Gamma^\beta_{\mu\nu}
\end{eqnarray*}
Und die Variation davon ist:
\begin{eqnarray*}
\delta R^\alpha_{\mu\nu\lambda} &=& -\delta\Gamma^\alpha_{\mu\nu,\lambda}+\delta\Gamma^\alpha_{\mu\lambda,\nu}\\
&&+\delta\Gamma^\alpha_{\beta\nu}\Gamma^\beta_{\mu\lambda}-\delta\Gamma^\alpha_{\beta\lambda}\Gamma^\beta_{\mu\nu}
+\Gamma^\alpha_{\beta\nu}\delta\Gamma^\beta_{\mu\lambda}-\Gamma^\alpha_{\beta\lambda}\delta\Gamma^\beta_{\mu\nu}
\end{eqnarray*}
Da aber $\delta\Gamma^\alpha_{\mu\nu}$ ein Tensor ist, können wir die kovariante Ableitung bilden:
\begin{eqnarray*}
\left(\delta\Gamma^\alpha_{\mu\nu}\right)_{;\lambda} &=& \left(\delta\Gamma^\alpha_{\mu\nu}\right)_{,\lambda}
+\delta\Gamma^\beta_{\mu\nu}\Gamma^\alpha_{\beta\lambda}-\delta\Gamma^\alpha_{\beta\nu}\Gamma^\beta_{\mu\lambda}-\delta\Gamma^\alpha_{\mu\beta}\Gamma^\beta_{\nu\lambda}\\
\delta\Gamma^\alpha_{\mu\nu,\lambda} &=& \left(\delta\Gamma^\alpha_{\mu\nu}\right)_{;\lambda}
-\delta\Gamma^\beta_{\mu\nu}\Gamma^\alpha_{\beta\lambda}+\delta\Gamma^\alpha_{\beta\nu}\Gamma^\beta_{\mu\lambda}+\delta\Gamma^\alpha_{\mu\beta}\Gamma^\beta_{\nu\lambda}\\
\left(\delta\Gamma^\alpha_{\mu\lambda}\right)_{;\nu} &=& \left(\delta\Gamma^\alpha_{\mu\lambda}\right)_{,\nu}
+\delta\Gamma^\beta_{\mu\lambda}\Gamma^\alpha_{\beta\nu}-\delta\Gamma^\alpha_{\beta\lambda}\Gamma^\beta_{\mu\nu}-\delta\Gamma^\alpha_{\mu\beta}\Gamma^\beta_{\lambda\nu}\\
\delta\Gamma^\alpha_{\mu\lambda,\nu} &=& \left(\delta\Gamma^\alpha_{\mu\lambda}\right)_{;\nu}
-\delta\Gamma^\beta_{\mu\lambda}\Gamma^\alpha_{\beta\nu}+\delta\Gamma^\alpha_{\beta\lambda}\Gamma^\beta_{\mu\nu}+\delta\Gamma^\alpha_{\mu\beta}\Gamma^\beta_{\lambda\nu}
\end{eqnarray*}
Oben eingesetzt ergibt dies:
\begin{eqnarray*}
\delta R^\alpha_{\mu\nu\lambda} &=& -\delta\Gamma^\alpha_{\mu\nu;\lambda}+\delta\Gamma^\alpha_{\mu\lambda;\nu}\\
&&-\left(-\delta\Gamma^\beta_{\mu\nu}\Gamma^\alpha_{\beta\lambda}+\delta\Gamma^\alpha_{\beta\nu}\Gamma^\beta_{\mu\lambda}+\delta\Gamma^\alpha_{\mu\beta}\Gamma^\beta_{\nu\lambda}\right)\\
&&+\left(-\delta\Gamma^\beta_{\mu\lambda}\Gamma^\alpha_{\beta\nu}+\delta\Gamma^\alpha_{\beta\lambda}\Gamma^\beta_{\mu\nu}+\delta\Gamma^\alpha_{\mu\beta}\Gamma^\beta_{\lambda\nu}\right)\\
&&+\delta\Gamma^\alpha_{\beta\nu}\Gamma^\beta_{\mu\lambda}-\delta\Gamma^\alpha_{\beta\lambda}\Gamma^\beta_{\mu\nu}
+\Gamma^\alpha_{\beta\nu}\delta\Gamma^\beta_{\mu\lambda}-\Gamma^\alpha_{\beta\lambda}\delta\Gamma^\beta_{\mu\nu}\\
&=&-\delta\Gamma^\alpha_{\mu\nu;\lambda}+\delta\Gamma^\alpha_{\mu\lambda;\nu}
\end{eqnarray*}
Dieses Ergebnis erhält man auch einfacher, indem man ohne Einschränkung der Allgemeinheit annimmt, die Koordinaten seien  am Punkt $x$ Normalkoordinaten. Dann verschwinden dort die $\Gamma$ und wir erhalten:
\begin{eqnarray*}
\delta R^\alpha_{\mu\nu\lambda} &=& -\delta\Gamma^\alpha_{\mu\nu,\lambda}+\delta\Gamma^\alpha_{\mu\lambda,\nu}\\
&&+\delta\Gamma^\alpha_{\beta\nu}\Gamma^\beta_{\mu\lambda}-\delta\Gamma^\alpha_{\beta\lambda}\Gamma^\beta_{\mu\nu}
+\Gamma^\alpha_{\beta\nu}\delta\Gamma^\beta_{\mu\lambda}-\Gamma^\alpha_{\beta\lambda}\delta\Gamma^\beta_{\mu\nu}\\
&=& -\delta\Gamma^\alpha_{\mu\nu,\lambda}+\delta\Gamma^\alpha_{\mu\lambda,\nu}
\end{eqnarray*}
Da $\delta\Gamma^\alpha_{\mu\nu}$ aber, wie oben gezeigt, ein Tensor ist, gilt in Normalkoordinaten:
\begin{eqnarray*}
\delta R^\alpha_{\mu\nu\lambda} &=& -\delta\Gamma^\alpha_{\mu\nu,\lambda}+\delta\Gamma^\alpha_{\mu\lambda,\nu}
=-\delta\Gamma^\alpha_{\mu\nu;\lambda}+\delta\Gamma^\alpha_{\mu\lambda;\nu}
\end{eqnarray*}
Letzteres ist aber eine Tensor-Gleichung und gilt unabhängig vom Koordinatensystem.
Die Variation des Ricci-Tensors erhält man durch Kontraktion:
\begin{eqnarray*}
\delta R_{\mu\lambda}= \delta R^\nu_{\mu\nu\lambda}&=& -\delta\Gamma^\nu_{\mu\nu;\lambda}+\delta\Gamma^\nu_{\mu\lambda;\nu}
\end{eqnarray*}
und schließlich die Variation des Krümmungsskalars:
\begin{eqnarray*}
R &=& R_{\mu\lambda}g^{\mu\lambda}\\
\delta R &=&\left( \delta R_{\mu\lambda}\right)g^{\mu\lambda}+R_{\mu\lambda}\delta g^{\mu\lambda}\\
&=&\left( \delta R_{\mu\lambda}\right)g^{\mu\lambda}-R_{\mu\lambda}\delta g_{\alpha\beta}g^{\mu\alpha}g^{\lambda\beta}\\
&=&\left(-\delta\Gamma^\nu_{\mu\nu;\lambda}+\delta\Gamma^\nu_{\mu\lambda;\nu}\right)g^{\mu\lambda}-R_{\mu\lambda}\delta g_{\alpha\beta}g^{\mu\alpha}g^{\lambda\beta}
\end{eqnarray*}
Da die kovariante Ableitung des metrischen Tensors $g^{\mu\lambda}$ verschwindet, kann der erste Ausdruck in eine Divergenz überführt werden:
\begin{eqnarray*}
\left(-\delta\Gamma^\nu_{\mu\nu;\lambda}+\delta\Gamma^\nu_{\mu\lambda;\nu}\right)g^{\mu\lambda}&=&\left(-g^{\mu\lambda}\delta\Gamma^\nu_{\mu\nu}+g^{\mu\nu}\delta\Gamma^\lambda_{\mu\nu}\right)_{;\lambda}
\end{eqnarray*}
Nun kann wegen \ref{riemann_gauss} der Satz von Gauß angewandt werden, denn
\begin{eqnarray*}
\sqrt{-g}F^\mu_{;\mu} &=& \left(\sqrt{-g}F^\mu\right)_{,\mu}
\end{eqnarray*}
Daher verschwindet das Integral über den ersten Ausdruck (bis auf Rand-Terme) und kann deswegen fallen gelassen werden.
Wir schreiben also:
\begin{eqnarray*}
\delta R &=&-R_{\mu\lambda}\delta g_{\alpha\beta}g^{\mu\alpha}g^{\lambda\beta}
\end{eqnarray*}
Es bleibt die Variation $\delta\sqrt{-g}$ zu bestimmen:
\begin{eqnarray*}
\delta \sqrt{-g} &=& \frac{1}{2}\frac{1}{\sqrt{-g}}\delta(-g)\\
-g-\delta g &=& -\det(g_{\mu\nu}+\delta g_{\mu\nu})\\
&=&  -\det\left(g_{\mu\sigma}\left(\delta^\sigma_\nu+g^{\sigma\lambda}\delta g_{\lambda\nu}\right)\right)\\
&=& -\det(g)\det\left(\delta^\sigma_\nu+g^{\sigma\lambda}\delta g_{\lambda\nu}\right)\\
&=& -g\left(1+g^{\lambda\nu}\delta g_{\lambda\nu}\right)\\
\delta(-g) &=& -gg^{\lambda\nu}\delta g_{\lambda\nu}\\
\delta \sqrt{-g} &=& \frac{1}{2}\sqrt{-g}g^{\lambda\nu}\delta g_{\lambda\nu}
\end{eqnarray*}
Die Gesamtvariation ist dann:
\begin{eqnarray*}
\delta(\sqrt{-g}R) &=& \left(\delta(\sqrt{-g}\right)R+\sqrt{-g}\delta R\\
&=& \sqrt{-g}\left(\frac{1}{2}g^{\lambda\nu}\delta g_{\lambda\nu}R-R_{\mu\lambda}\delta g_{\alpha\beta}g^{\mu\alpha}g^{\lambda\beta}\right)\\
&=& \sqrt{-g}\left(\frac{1}{2}g^{\mu\nu}R-R^{\mu\nu}\right)\delta g_{\mu\nu}\\
&=& -\sqrt{-g}G^{\mu\nu}\delta g_{\mu\nu}\\
\delta\int\sqrt{-g}R\ d^4x &=& -\int \sqrt{-g}G^{\mu\nu}\delta g_{\mu\nu}\ d^4x
\end{eqnarray*}
Die Konstante $c$ in \ref{einst_hilbert} ist also $-1$. Der Vollständigkeit halber sei hier der Gesamtausdruck für die Variation gegeben:
\begin{eqnarray}
\label{R_variation}
\delta(\sqrt{-g}R) &=& -\sqrt{-g}G^{\mu\nu}\delta g_{\mu\nu}+\left(\sqrt{-g}\left(-g^{\mu\lambda}\delta\Gamma^\nu_{\mu\nu}+g^{\mu\nu}\delta\Gamma^\lambda_{\mu\nu}\right)\right)_{,\lambda}
\end{eqnarray}
\end{subsection}
\eeeccc

\beginchap{Energie-Impuls-Tensor in der speziellen Relativitätstheorie anhand zweier Beispiele}
\begin{subsection}{Skalares Feld}
Wir untersuchen zunächst den allgemeinen Fall eines skalaren Feldes $\phi$, und eine Lagrangedichte $\mathcal{L}(\phi, \phi_{,\mu})$ welche nur von den ersten Ableitungen abhängt. Die Bewegungsgleichungen lauten dann
\begin{eqnarray*}
\frac{\partial \mathcal{L}}{\partial \phi}-\frac{\partial \mathcal{L}}{\partial \phi_{,\mu}} &=& 0
\end{eqnarray*}
Für ein Feld der 'Masse' $m$ ist die Lagrangedichte
\begin{eqnarray}
\label{lagr_scalar}
\frac{1}{2}\left(\phi_{,\mu}\phi_{,\nu}\eta^{\mu\nu}+m^2\phi\phi\right)
\end{eqnarray}
und damit die Bewegungsgleichungen
\begin{eqnarray}
\label{scalar_motion}
F &=& m^2\phi-\phi_{,\mu\nu}\eta^{\mu\nu} = 0
\end{eqnarray}
Der allgemeine kanonische Energie-Impuls-Tensor lautet
\begin{eqnarray*}
\Theta^\mu_\lambda &=& \frac{\partial \mathcal{L}}{\partial \phi_{,\mu}}\phi_{,\lambda}-\mathcal{L}\delta^\mu_\lambda
\end{eqnarray*}
Für unseren konkreten Fall heißt das:
\begin{eqnarray*}
\Theta^\mu_\lambda &=& \phi_{,\nu}\eta^{\nu\mu}\phi_{,\lambda}-\mathcal{L}\delta^\mu_\lambda
\end{eqnarray*}
Durch hochziehen von $\lambda$ entsteht ein symmetrischer Tensor:
\begin{eqnarray}
\label{scalar_energy_momentum_tensor}
T^{\mu\nu} &=& \Theta^\mu_\lambda\eta^{\lambda\nu} = \phi_{,\alpha}\eta^{\alpha\mu}\phi_{,\lambda}\eta^{\lambda\nu}-\mathcal{L}\eta^{\mu\nu}
\end{eqnarray}
\end{subsection}

\begin{subsection}{Elektromagnetisches Feld}
Das elektromagnetische Feld hat die Lagrange-Dichte
\begin{eqnarray}
\label{electromagn_lagr}
\mathcal{L}(A_{\mu,\nu}) &=& \frac{1}{4}F_{\mu\nu}F_{\alpha\beta}\eta^{\mu\alpha}\eta^{\nu\beta}\\
F_{\mu\nu} &=& A_{\mu,\nu}-A_{\nu,\mu}
\end{eqnarray}
Und die Bewegungsgleichung lautet:
\begin{eqnarray}
\label{maxwell_affine}
\left(\frac{\partial \mathcal{L}}{\partial A_{\mu,\nu}}\right)_{,\nu} &=& F^{\mu\nu}_{,\nu}=-F^\mu=0
\end{eqnarray}
Und für den kanonischen Energie-Impuls-Tensor erhalten wir
\begin{eqnarray}
\label{em_canonic_energy_momentum}
\Theta^\mu_\lambda &=& \frac{\partial \mathcal{L}}{\partial A_{\alpha,\mu}}A_{\alpha,\lambda}-\mathcal{L}\delta^\mu_\lambda\\
&=& F_{\beta\gamma}\eta^{\alpha\beta}\eta^{\gamma\mu}A_{\alpha,\lambda}-\mathcal{L}\delta^\mu_\lambda
\end{eqnarray}
Durch hochziehen ergibt das
\begin{eqnarray*}
\Theta^{\mu\nu} &=& F_{\beta\gamma}\eta^{\alpha\beta}\eta^{\gamma\mu}A_{\alpha,\lambda}\eta^{\lambda\nu}-\mathcal{L}\eta^{\mu\nu}
\end{eqnarray*}
einen asymmetrischen Tensor. Hingegen ist
\begin{eqnarray}
\label{em_energy_momentum}
T^{\mu\nu} &=& F_{\beta\gamma}\eta^{\alpha\beta}\eta^{\gamma\mu}F_{\alpha\lambda}\eta^{\lambda\nu}-\mathcal{L}\eta^{\mu\nu}
=F^{\nu\alpha}F^{\nu\beta}\eta_{\alpha\beta}-\mathcal{L}\eta^{\mu\nu}
\end{eqnarray}
sehr wohl symmetrisch.
Nun lässt sich aber zeigen, dass der Symmetrisierungs-Term $\Delta^{\nu\sigma}=F^{\mu\nu}A_{\lambda,\mu}\eta^{\lambda\sigma}$ folgende Eigenschaften hat:
\begin{eqnarray}
\label{em_symmetry_term}
\Delta^{\nu\sigma}_{,\nu} &=& 0\\
\int \Delta^{0\sigma}\ d^3x &=& 0
\end{eqnarray}
\begin{proof}
Zunächst gilt wegen den Bewegungsgleichungen $F^{\mu\nu}_{,\mu}=0$:
\begin{eqnarray*}
\Delta^{\nu\sigma}&=&F^{\mu\nu}A_{\lambda,\mu}\eta^{\lambda\sigma}\\
&=&\left(F^{\mu\nu}A_{\lambda}\eta^{\lambda\sigma}\right)_{,\mu}-F^{\mu\nu}_{,\mu}A_{\lambda}\eta^{\lambda\sigma}=\left(F^{\mu\nu}A_{\lambda}\eta^{\lambda\sigma}\right)_{,\mu}
\end{eqnarray*}
Die erste Behauptung gilt, da
\begin{eqnarray*}
\Delta^{\nu\sigma}_{,\nu} &=& \left(F^{\mu\nu}A_{\lambda}\eta^{\lambda\sigma}\right)_{,\mu\nu} = 0
\end{eqnarray*}
wegen der Antisymmetrie von $F^{\mu\nu}$. Die zweite Behauptung gilt
\begin{eqnarray*}
\int \Delta^{0\sigma}\ d^3x &=& \int \left(F^{\mu 0}A_{\lambda}\eta^{\lambda\sigma}\right)_{,\mu}\ d^3x = \int \left(F^{00}A_{\lambda}\eta^{\lambda\sigma}\right)_{,0}\ d^3x+\int \left(F^{k0}A_{\lambda}\eta^{\lambda\sigma}\right)_{,k}\ d^3x
\end{eqnarray*}
da das erste Integral verschwindet denn $F^{00}=0$ und das zweite verschwindet, weil die Randterme verschwinden.
\qed
\end{proof}
Daher erhalten wir:
\begin{eqnarray*}
\Theta^{\nu\sigma}_{,\nu} &=& T^{\nu\sigma}_{,\nu} = 0\\
P^\sigma=\int \Theta^{0\sigma}\ d^3x &=& \int T^{0\sigma}\ d^3x 
\end{eqnarray*}
Das heißt, die globalen Werte für Energie $P^0$ und Impuls $P^k$ sind für $\Theta$ und $T$ gleich.
\end{subsection}
\eeeccc

\beginchap{Das komplette Variatiations-Problem}
\begin{subsection}{Grundsätzliches}
Wir widmen uns nun dem Variationproblem zur Wirkung \ref{lagrange_total}
\begin{eqnarray*}
S &=& \int dx^0dx^1dx^2dx^3\ \sqrt{-g}\left(\mathcal{L}-\frac{1}{2\kappa}R\right)
\end{eqnarray*}
Dabei steht die Struktur von $\mathcal{L}$ noch nicht fest. Um aber zu Resultaten zu gelangen, die von der Wahl der Koordinaten unabhängig sind, sollte $\mathcal{L}$ ein echter Skalar sein. Wir haben mit \ref{R_variation} bereits die Variation von $\sqrt{-g}R$ berechnet:
\begin{eqnarray*}
\delta(\sqrt{-g}R) &=& -\sqrt{-g}G^{\mu\nu}\delta g_{\mu\nu}+\left(\sqrt{-g}\left(-g^{\mu\lambda}\delta\Gamma^\nu_{\mu\nu}+g^{\mu\nu}\delta\Gamma^\lambda_{\mu\nu}\right)\right)_{,\lambda}
\end{eqnarray*}
Es bleibt einen Ausdruck für $\delta(\sqrt{-g}\mathcal{L})$ zu finden. Diesen wollen wir für die obigen Beispiele ermitteln.
Allgemein gilt
\begin{eqnarray*}
\delta(\sqrt{-g}\mathcal{L}) &=& \sqrt{-g}\left(\frac{1}{2}g^{\mu\nu}\delta g_{\mu\nu}\mathcal{L}+\delta \mathcal{L}\right)
\end{eqnarray*}
\end{subsection}

\begin{subsection}{Skalares Feld der Masse $m$}
Wir rekapitulieren die Lagrangedichte \ref{lagr_scalar}
\begin{eqnarray*}
\mathcal{L} &=& \frac{1}{2}\left(\phi_{,\mu}\phi_{,\nu}g^{\mu\nu}+m^2\phi\phi\right)
\end{eqnarray*}
Wobei wir $\eta^{\mu\nu}$ durch $g^{\mu\nu}$ ersetzt haben und dadurch einen allgemein kovarianten Ausdruck erhalten. Wir berechnen
\begin{eqnarray*}
\delta \mathcal{L} &=& \frac{1}{2}\phi_{,\mu}\phi_{,\nu}\delta g^{\mu\nu}+\phi_{,\mu}g^{\mu\nu}\delta\phi_{,\nu}+m^2\phi\delta\phi\\
&=& -\frac{1}{2}\phi_{,\mu}\phi_{,\nu}g^{\mu\alpha}g^{\nu\beta}\delta g_{\alpha\beta}+\phi_{,\mu}g^{\mu\nu}\delta\phi_{,\nu}+m^2\phi\delta\phi
\end{eqnarray*}
und erhalten
\begin{eqnarray*}
\delta (\sqrt{-g}\mathcal{L}) &=& \sqrt{-g}\left(\frac{1}{2}g^{\mu\nu}\delta g_{\mu\nu}\mathcal{L}-\frac{1}{2}\phi_{,\mu}\phi_{,\nu}g^{\mu\alpha}g^{\nu\beta}\delta g_{\alpha\beta}+\phi_{,\mu}g^{\mu\nu}\delta\phi_{,\nu}+m^2\phi\delta\phi\right)\\
&=& \sqrt{-g}\left(-\frac{1}{2}\left(\phi_{,\mu}\phi_{,\nu}g^{\mu\alpha}g^{\nu\beta}-g^{\alpha\beta}\mathcal{L}\right)\delta g_{\alpha\beta}+\phi_{,\mu}g^{\mu\nu}\delta\phi_{,\nu}+m^2\phi\delta\phi\right)\\
&=& \sqrt{-g}\left(-\frac{1}{2}T^{\alpha\beta}\delta g_{\alpha\beta}+\phi_{,\mu}g^{\mu\nu}\delta\phi_{,\nu}+m^2\phi\delta\phi\right)
\end{eqnarray*}
Wobei hier
\begin{eqnarray*}
T^{\alpha\beta} &=& \phi_{,\mu}\phi_{,\nu}g^{\mu\alpha}g^{\nu\beta}-g^{\alpha\beta}\mathcal{L}
\end{eqnarray*}
den symmetrische Energie-Impuls-Tensor \ref{scalar_energy_momentum_tensor} in allgemein kovarianter Form bedeudet.
Nun widmen wir uns dem Term $\phi_{,\mu}g^{\mu\nu}\delta\phi_{,\nu}$:
\begin{eqnarray*}
\sqrt{-g}\phi_{,\mu}g^{\mu\nu}\delta\phi_{,\nu} &=& \sqrt{-g}\left(\left(\phi_{,\mu}g^{\mu\nu}\delta\phi\right)_{;\nu}-\left(\phi_{,\mu}g^{\mu\nu}\right)_{;\nu}\delta\phi\right)\\
&=& \left(\sqrt{-g}\phi_{,\mu}g^{\mu\nu}\delta\phi\right)_{,\nu}+\sqrt{-g}\left(-\left(\phi_{,\mu}g^{\mu\nu}\right)_{;\nu}\delta\phi\right)
\end{eqnarray*}
und erhalten
\begin{eqnarray*}
\delta (\sqrt{-g}\mathcal{L}) &=& \sqrt{-g}\left(-\frac{1}{2}T^{\alpha\beta}\delta g_{\alpha\beta}-\left(\phi_{,\mu}g^{\mu\nu}\right)_{;\nu}\delta\phi+m^2\phi\delta\phi\right)+\left(\sqrt{-g}\phi_{,\mu}g^{\mu\nu}\delta\phi\right)_{,\nu}\\
&=& \sqrt{-g}\left(-\frac{1}{2}T^{\alpha\beta}\delta g_{\alpha\beta}+\left(m^2\phi-\left(\phi_{,\mu}g^{\mu\nu}\right)_{;\nu}\right)\delta\phi\right)+\left(\sqrt{-g}\phi_{,\mu}g^{\mu\nu}\delta\phi\right)_{,\nu}\\
&=& \sqrt{-g}\left(-\frac{1}{2}T^{\alpha\beta}\delta g_{\alpha\beta}+F\delta\phi\right)+\left(\sqrt{-g}\phi_{,\mu}g^{\mu\nu}\delta\phi\right)_{,\nu}
\end{eqnarray*}
mit der allgemein kovarianten Variante der Bewegungsgleichung \ref{scalar_motion}
\begin{eqnarray*}
F &=& m^2\phi-\left(\phi_{,\mu}g^{\mu\nu}\right)_{;\nu}= m^2\phi-\phi_{;\mu\nu}g^{\mu\nu} =0
\end{eqnarray*}
\end{subsection}


\begin{subsection}{Elektromagnetisches Feld}
Wir gehen von \ref{electromagn_lagr} aus
\begin{eqnarray*}
\mathcal{L} &=& \frac{1}{4}F_{\mu\nu}F_{\alpha\beta}g^{\mu\alpha}g^{\nu\beta}\\
F_{\mu\nu} &=& A_{\mu;\nu}-A_{\nu;\mu}=A_{\mu,\nu}-A_{\nu,\mu}
\end{eqnarray*}
und Rechnen
\begin{eqnarray*}
\delta\mathcal{L} &=& \frac{1}{2}F_{\mu\nu}F_{\alpha\beta}g^{\mu\alpha}\delta g^{\nu\beta}+F_{\mu\nu}g^{\mu\alpha}g^{\nu\beta}\delta A_{\alpha;\beta}\\
&=& -\frac{1}{2}F_{\mu\nu}F_{\alpha\beta}g^{\mu\alpha}g^{\nu\sigma}g^{\beta\tau}\delta g_{\sigma\tau}+\left(F_{\mu\nu}g^{\mu\alpha}g^{\nu\beta}\delta A_\alpha\right)_{;\beta}-\left(F_{\mu\nu}g^{\mu\alpha}g^{\nu\beta}\right)_{;\beta}\delta A_\alpha\\
&=& -\frac{1}{2}F^{\mu\alpha}F^{\nu\beta}g_{\alpha\beta}\delta g_{\mu\nu}-F^{\mu\nu}_{;\nu}\delta A_\mu+\left(F^{\mu\nu}\delta A_\mu\right)_{;\nu}\\
\delta\left(\sqrt{-g}\mathcal{L}\right) &=&\sqrt{-g}\left(\frac{1}{2}\left(g^{\mu\nu}\mathcal{L}-F^{\mu\alpha}F^{\nu\beta}g_{\alpha\beta}\right)\delta g_{\mu\nu}-F^{\mu\nu}_{;\nu}\delta A_\mu\right)+\left(\sqrt{-g}F^{\mu\nu}\delta A_\mu\right)_{,\nu}\\
&=&\sqrt{-g}\left(-\frac{1}{2}T^{\mu\nu}\delta g_{\mu\nu}+F^\mu\delta A_\mu\right)+\left(\sqrt{-g}F^{\mu\nu}\delta A_\mu\right)_{,\nu}
\end{eqnarray*}
Mit
\begin{eqnarray*}
T^{\mu\nu} &=& F^{\mu\alpha}F^{\nu\beta}g_{\alpha\beta}-g^{\mu\nu}\mathcal{L}
\end{eqnarray*}
, der allgemein kovarianten Version des symmetrisierten Energie-Impuls-Tensors \ref{em_energy_momentum} und
\begin{eqnarray*}
F^\mu &=& -F^{\mu\nu}_{;\nu}
\end{eqnarray*}
, der allgemein kovarianten Version der Bewegungsgleichungen (Vakuum-Maxwell-Gleichungen) \ref{maxwell_affine}
\end{subsection}

\begin{subsection}{Allgemeines Schema}
Wir vermuten nun, dass allgemein gilt
\begin{eqnarray*}
\delta \left(\sqrt{-g}\mathcal{L}\right)&=& \sqrt{-g}\left(-\frac{1}{2}T^{\mu\nu}\delta g_{\mu\nu}+F\delta\phi\right)+X^\mu_{,\mu}
\end{eqnarray*}
und das somit \ref{total_variation_integral} 
\begin{eqnarray*}
\delta S &=& \int dx^0dx^1dx^2dx^3\ \sqrt{-g}\left(\frac{1}{2}\left(-T^{\mu\nu}+\frac{1}{\kappa}G^{\mu\nu}\right)\delta g_{\mu\nu}+F\delta\phi\right)
\end{eqnarray*}
für die totale Wirkung \ref{lagrange_total}
\begin{eqnarray*}
S &=& \int dx^0dx^1dx^2dx^3\ \sqrt{-g}\left(\mathcal{L}-\frac{1}{2\kappa}R\right)
\end{eqnarray*}
Weiter ist
\begin{eqnarray}
\label{abstract_total_variation}
\delta \left(\sqrt{-g}\left(\mathcal{L}-\frac{1}{2\kappa}R\right)\right) &=&\sqrt{-g}\left(\frac{1}{2}\left(-T^{\mu\nu}+\frac{1}{\kappa}G^{\mu\nu}\right)\delta g_{\mu\nu}+F\delta\phi\right)\cr
&&-\frac{1}{2\kappa}\left(\sqrt{-g}\left(-g^{\mu\lambda}\delta\Gamma^\nu_{\mu\nu}+g^{\mu\nu}\delta\Gamma^\lambda_{\mu\nu}\right)\right)_{,\lambda}+X^\mu_{,\mu}\cr
&=&\sqrt{-g}\left(\frac{1}{2}\left(-T^{\mu\nu}+\frac{1}{\kappa}G^{\mu\nu}\right)\delta g_{\mu\nu}+F\delta\phi\right)\cr
&&\left(-\frac{1}{2\kappa}\left(\sqrt{-g}\left(-g^{\mu\lambda}\delta\Gamma^\nu_{\mu\nu}+g^{\mu\nu}\delta\Gamma^\lambda_{\mu\nu}\right)\right)+X^\lambda\right)_{,\lambda}
\end{eqnarray}
wobei wir \ref{R_variation} benutzt haben. $F\delta\phi$ ist abstrakt zu verstehen und hängt genau wie $X$ von der kontreten Struktur von $\mathcal{L}$ ab.
\end{subsection}
\eeeccc

\beginchap{Globale Erhaltungsgrößen in der allgemeinen Relativitätstheorie}
\begin{subsection}{Noethers Theorem in der allgemeinen Relativitätstheorie}
Unter der Bedingung, dass eine infinitesimale Transformation alle beteiligten Felder, also auch die Metrik transformiert:
\begin{eqnarray*}
g_{\mu\nu}(x) &\mapsto& g_{\mu\nu}(\epsilon,x)\\
\phi(x) &\mapsto& g\phi(\epsilon,x)
\end{eqnarray*}
und dass die globale Topologie asymptotisch flach ist, so dass ein einziges Koordinatensystem den ganzen Raum überdeckt, 
und dass zusätzlich
\begin{eqnarray*}
\frac{d}{d\epsilon}\left(\sqrt{-g}\left(\mathcal{L}-\frac{1}{2\kappa}R\right)\right|_{\epsilon=0} &=& D^\mu_{,\mu}
\end{eqnarray*}
 gilt, dann lässt sich eine globale Erhaltungsgröße definieren.
Wir definieren
\begin{eqnarray*}
\tilde{\mathcal{L}} &=& \sqrt{-g}\left(\mathcal{L}-\frac{1}{2\kappa}R\right)
\end{eqnarray*}
und schreiben \ref{abstract_total_variation}
\begin{eqnarray*}
\delta \tilde{\mathcal{L}}&=&\sqrt{-g}\left(\frac{1}{2}\left(-T^{\mu\nu}+\frac{1}{\kappa}G^{\mu\nu}\right)\delta g_{\mu\nu}+F\delta\phi\right)\cr
&&\left(-\frac{1}{2\kappa}\left(\sqrt{-g}\left(-g^{\mu\lambda}\delta\Gamma^\nu_{\mu\nu}+g^{\mu\nu}\delta\Gamma^\lambda_{\mu\nu}\right)\right)+X^\lambda\right)_{,\lambda}
\end{eqnarray*}
wobei $\delta$ nun die Bedeutung
\begin{eqnarray*}
\delta(Y) &=& \frac{d}{d\epsilon}\left(Y\right|_{\epsilon=0}
\end{eqnarray*}
hat. Unter der Voraussetzung, dass die Felder die Bewegungsgleichungen erfüllen, wird daraus
\begin{eqnarray*}
\delta \tilde{\mathcal{L}}&=&\left(-\frac{1}{2\kappa}\left(\sqrt{-g}\left(-g^{\mu\lambda}\delta\Gamma^\nu_{\mu\nu}+g^{\mu\nu}\delta\Gamma^\lambda_{\mu\nu}\right)\right)+X^\lambda\right)_{,\lambda} =D^\mu_{,\mu}
\end{eqnarray*}
oder
\begin{eqnarray*}
\left(-\frac{1}{2\kappa}\left(\sqrt{-g}\left(-g^{\mu\lambda}\delta\Gamma^\nu_{\mu\nu}+g^{\mu\nu}\delta\Gamma^\lambda_{\mu\nu}\right)\right)+X^\lambda-D^\lambda\right)_{,\lambda} = 0
\end{eqnarray*}
mit
\begin{eqnarray}
\label{gen_current}
j^\lambda &=& -\frac{1}{2\kappa}\left(\sqrt{-g}\left(-g^{\mu\lambda}\delta\Gamma^\nu_{\mu\nu}+g^{\mu\nu}\delta\Gamma^\lambda_{\mu\nu}\right)\right)+X^\lambda-D^\lambda
\end{eqnarray}
erhalten wir die globale Erhaltungsgröße
\begin{eqnarray*}
Q &=& \int dx^1dx^2dx^3\ j^0
\end{eqnarray*}
\end{subsection}

\begin{subsection}{Energie und Impuls in der allgemeinen Relativitätstheorie}
\begin{subsubsection}{kanonischer Energie-Impuls-Tensor nach Noether}
Für
\begin{eqnarray*}
\delta(Y) &=& Y_{,\tau}
\end{eqnarray*}
erhalten wir die Erhaltungssätze für Energie ($\tau=0$) und Impuls $P_k$ ($\tau=k$).
Um konkrete Ausdrücke zu erhalten, brauchen wir die $X^\lambda$. Für das skalare Feld heißt das
\begin{eqnarray*}
X^\lambda_\tau &=& \sqrt{-g}\phi_{,\mu}g^{\mu\lambda}\phi_{,\tau}=\sqrt{-g}\left(\Theta^\lambda_\tau+\mathcal{L}\delta^\lambda_\tau\right)
\end{eqnarray*}
und für das elektromagnetische Feld ist das
\begin{eqnarray*}
X^\lambda_\tau &=& \sqrt{-g}F^{\mu\lambda}A_{\mu,\tau}=\sqrt{-g}\left(\Theta^\lambda_\tau+\mathcal{L}\delta^\lambda_\tau\right)
\end{eqnarray*}
so erhalten wir den kanonischen Energie-Impuls-Tensor für das Gesamtsystem analog zu \ref{gen_current}:
\begin{eqnarray}
\label{canonic_tensor}
\mathfrak{T}^\lambda_\tau &=& -\frac{1}{2\kappa}\left(\sqrt{-g}\left(-g^{\mu\lambda}\Gamma^\nu_{\mu\nu,\tau}+g^{\mu\nu}\Gamma^\lambda_{\mu\nu,\tau}\right)\right)+X^\lambda_\tau-\tilde{\mathcal{L}}\delta^\lambda_\tau\cr
&=& \sqrt{-g}\left(\frac{1}{2\kappa}\left(g^{\mu\lambda}\Gamma^\nu_{\mu\nu,\tau}-g^{\mu\nu}\Gamma^\lambda_{\mu\nu,\tau}+R\delta^\lambda_\tau\right)+\Theta^\lambda_\tau\right)
\end{eqnarray}
und damit auch den Energie-Impuls-Vektor
\begin{eqnarray*}
P_\tau &=& \int dx^1dx^2dx^3\ \mathfrak{T}^0_\tau 
\end{eqnarray*}
\end{subsubsection}

\begin{subsubsection}{Verallgemeinerter kanonischer Enerie-Impuls-Tensor}
Der Tensor \ref{canonic_tensor} hängt über $\Theta^\lambda_\tau$ von der konkreten Form von $\mathcal{L}$, der Lagrange-'Dichte' für Materie und Felder ab. Im konkreten Fall skalarer Felder ist dieser gleich
\begin{eqnarray*}
\Theta^\lambda_\tau &=& T^{\lambda\nu}g_{\tau\nu}
\end{eqnarray*}
Wobei $T^{\lambda\nu}g$ der symmetrische Energie-Impuls-Tensor ist, wie er auch in die Einsteinschen Gleichungen
\begin{eqnarray*}
G^{\lambda\nu} &=& \kappa T^{\lambda\nu}
\end{eqnarray*}
eingeht. Daher ist es zumindest für diesen konkreten Fall legitim
\begin{eqnarray*}
\Theta^\lambda_\tau &=& \frac{1}{\kappa}G^\lambda_\tau
\end{eqnarray*}
zu setzen. Daher wird \ref{canonic_tensor}:
\begin{eqnarray}
\label{canonic_energy_momentum_density}
\mathfrak{T}^\lambda_\tau &=& \sqrt{-g}\left(\frac{1}{2\kappa}\left(g^{\mu\lambda}\Gamma^\nu_{\mu\nu,\tau}-g^{\mu\nu}\Gamma^\lambda_{\mu\nu,\tau}+R\delta^\lambda_\tau\right)+\frac{1}{\kappa}G^\lambda_\tau\right)\cr
&=& \frac{1}{\kappa}\sqrt{-g}\left(\frac{1}{2}\left(g^{\mu\lambda}\Gamma^\nu_{\mu\nu,\tau}-g^{\mu\nu}\Gamma^\lambda_{\mu\nu,\tau}+R\delta^\lambda_\tau\right)+G^\lambda_\tau\right)\cr
&=& \frac{1}{\kappa}\sqrt{-g}\left(\frac{1}{2}\left(g^{\mu\lambda}\Gamma^\nu_{\mu\nu,\tau}-g^{\mu\nu}\Gamma^\lambda_{\mu\nu,\tau}\right)+R^\lambda_\tau\right)
\end{eqnarray}
Wobei $R^\lambda_\tau$ der gemischte Ricci-Tensor ist. Für den Fall des skalaren Feldes geht die Divergenzfreiheit
\begin{eqnarray*}
\left(\frac{1}{\kappa}\sqrt{-g}\left(\frac{1}{2}\left(g^{\mu\lambda}\Gamma^\nu_{\mu\nu,\tau}-g^{\mu\nu}\Gamma^\lambda_{\mu\nu,\tau}\right)+R^\lambda_\tau\right)\right)_{,\lambda} &=& \mathfrak{T}^\lambda_{\tau,\lambda} =0
\end{eqnarray*}
direkt aus der Konstruktion mittels Noethers Theorem hervor. Diese gilt aber allgemein. Für den Fall des elektromagnetischen Feldes
hingegen haben wir den kanonischen Energie-Impuls-Tensor \ref{em_canonic_energy_momentum}
\begin{eqnarray*}
\Theta^\mu_\lambda &=& F^{\alpha\mu}A_{\alpha,\lambda}-\mathcal{L}\delta^\mu_\lambda
\end{eqnarray*}
Der swymmetrische Energie-Impuls-Tensor, der auch in die Einsteinsche Gleichung $G^{\mu\nu}=\kappa T^{\mu\nu}$ eingeht, lautet in gemischter Form:
\begin{eqnarray*}
T^\mu_\lambda &=& F^{\alpha\mu}F_{\alpha\lambda}-\mathcal{L}\delta^\mu_\lambda= F^{\alpha\mu}\left(A_{\alpha;\lambda}-A_{\lambda;\alpha}\right)-\mathcal{L}\delta^\mu_\lambda
= F^{\alpha\mu}\left(A_{\alpha,\lambda}-A_{\lambda,\alpha}\right)-\mathcal{L}\delta^\mu_\lambda
\end{eqnarray*}
Die Differenz
\begin{eqnarray*}
\Delta^\mu_\lambda &=& \Theta^\mu_\lambda-T^\mu_\lambda\\
&=& F^{\alpha\mu}A_{\lambda,\alpha}
\end{eqnarray*}
erfüllt die Bedingungen 
\begin{eqnarray*}
\left(\sqrt{-g}\Delta^\mu_\lambda\right)_{,\mu} &=& 0\\
\int \sqrt{-g}\Delta^0_\lambda\ d^3x &=& 0
\end{eqnarray*}
erfüllen.
\begin{proof}
Wir formen um:
\begin{eqnarray*}
\sqrt{-g}\Delta^\mu_\lambda &=& \sqrt{-g}F^{\alpha\mu}A_{\lambda,\alpha}\\
&=& \left(\sqrt{-g}F^{\alpha\mu}A_\lambda\right)_{,\alpha}
-\sqrt{-g}\left(\frac{1}{2}g^{\sigma\tau}g_{\sigma\tau,\alpha}F^{\alpha\mu}A_\lambda+F^{\alpha\mu}_{,\alpha}A_\lambda\right)
\end{eqnarray*}
Nun gelten die Vakuum-Maxwell-Gleichungen $F^{\alpha\mu}_{;\alpha}=0$:
\begin{eqnarray*}
F^{\alpha\beta}_{;\alpha} &=& F^{\alpha\beta}_{,\alpha}+\Gamma^\alpha_{\alpha\sigma}F^{\sigma\beta}+\Gamma^\beta_{\alpha\sigma}F^{\alpha\sigma}\\
&=&F^{\alpha\beta}_{,\alpha}+\Gamma^\alpha_{\alpha\sigma}F^{\sigma\beta}\\
&=&F^{\alpha\beta}_{,\alpha}+\frac{1}{2}g^{\sigma\tau}g_{\sigma\tau,\alpha}F^{\alpha\beta}\\
F^{\alpha\beta}_{,\alpha}&=&F^{\alpha\beta}_{;\alpha}-\frac{1}{2}g^{\sigma\tau}g_{\sigma\tau,\alpha}F^{\alpha\beta}\\
&=&-\frac{1}{2}g^{\sigma\tau}g_{\sigma\tau,\alpha}F^{\alpha\beta}\\
F^{\alpha 0}_{,\alpha}A_\lambda&=&-\frac{1}{2}g^{\sigma\tau}g_{\sigma\tau,\alpha}F^{\alpha 0}A_\lambda
\end{eqnarray*}
Das heißt es gilt
\begin{eqnarray*}
\sqrt{-g}\Delta^\mu_\lambda &=& \left(\sqrt{-g}F^{\alpha\mu}A_\lambda\right)_{,\alpha}
\end{eqnarray*}
Womit sich die beiden zu beweisenden Bedingungen ergeben:
\begin{eqnarray*}
\left(\sqrt{-g}\Delta^\mu_\lambda\right)_{,\mu} &=& \left(\sqrt{-g}F^{\alpha\mu}A_\lambda\right)_{,\alpha\mu}=0
\end{eqnarray*}
denn $F^{\alpha\mu}$ ist antisymmetrisch.
\begin{eqnarray*}
\int \sqrt{-g}\Delta^0_\lambda dx^1dx^2dx^3 &=& \int \left(\sqrt{-g}F^{\alpha 0}A_\lambda\right)_{,\alpha} dx^1dx^2dx^3\\
&=& \int \left(\sqrt{-g}F^{k0}A_\lambda\right)_{,k} dx^1dx^2dx^3 = 0
\end{eqnarray*}
Wobei $k=1,2,3$ nur über die drei Raumkoordinaten läuft, denn $F^{00}=0$. Daher haben wir ein Integral über eine Divergenz. Da wir aber annehmen, dass $F^{k0}$ im unendlichen verschwinden, ergibt das Integral Null.
\qed
\end{proof}
Die Tensor-Dichte \ref{canonic_energy_momentum_density} beschreibt in allen Fällen vier Erhaltungsgrössen
\begin{eqnarray*}
P_\lambda &=& \int \sqrt{-g}T^0_\lambda dx^1dx^2dx^3 =\int \mathfrak{T}^0_\lambda dx^1dx^2dx^3 
\end{eqnarray*}

\end{subsubsection}

\begin{subsubsection}{Landau-Lifschitz Pseudotensor}
\begin{paragraph}{Definition}
Es gibt eine berühmter Ansatz, der einen symmetrischen (pseudo)-Tensor für den totalen Energie-Impuls-Vektor liefert.
\begin{eqnarray*}
t^{\mu\nu} &=& -\frac{1}{\kappa}G^{\mu\nu}+\frac{1}{2\kappa(-g)}\left((-g)\left(g^{\mu\nu}g^{\alpha\beta}-g^{\alpha\mu}g^{\beta\nu}\right)\right)_{,\alpha\beta}
\end{eqnarray*}
$t^{\mu\nu}$ hat folgende Eigenschaften:
\begin{enumerate}
	\item $((-g)(t^{\mu\nu}+T^{\mu\nu}))_{,\mu}=0$.
	\item $t^{\mu\nu}$ enthält keine zweiten Ableitungen der $g_{\mu\nu,\alpha\beta}$.
	\item $t^{\mu\nu}=t^{\nu\mu}$.
	\item $t_{\mu\nu}$ wird ausschließlich durch $g_{\mu\nu}$ und deren erste Ableitungen gebildet.
\end{enumerate}
Dadurch kann ein totaler Energie-Impuls-Vektor definiert werden, dessen Zeit-Ableitung verschwindet:
\begin{eqnarray*}
P^ \mu &=& \int dx^1dx^2dx^3\ (-g)(t^{\mu 0}+T^{\mu 0})
\end{eqnarray*}
\end{paragraph}

\begin{paragraph}{$t^{\mu\nu}$ enthält keine zweiten Ableitungen}
Im Abschnitt \ref{tens2_terms} sind alle Terme von der Form $X_{\mu\lambda}$ die aus ersten und zweiten Ableitungen bestehen aufgelistet.
Da wir uns nur für zweite Ableitungen interessieren, konnen nur die folgenden 4 Terme in Frage:
\begin{eqnarray*}
a^1_{\mu\lambda} &=& g^{\sigma\tau}g_{\mu\lambda,\sigma\tau}\\
a^2_{\mu\lambda} &=& g^{\sigma\tau}g_{\sigma\tau,\mu\lambda}\\
b^1_{\mu\lambda} &=& g^{\sigma\tau}g_{\mu\sigma,\lambda\tau}\\
b^2_{\mu\lambda} &=& g^{\sigma\tau}g_{\lambda\sigma,\mu\tau}=b^1_{\lambda\mu}
\end{eqnarray*}
Ferner gibt es zwei skalare Terme die zweite Ableitungen enthalten:
\begin{eqnarray*}
A &=& g^{\alpha\beta}g^{\mu\nu}g_{\alpha\beta,\mu\nu}\\
B &=& g^{\alpha\mu}g^{\beta\nu}g_{\alpha\beta,\mu\nu}
\end{eqnarray*}
die nach Multiplikation mit $g_{\mu\lambda}$ in die selbe Form gebracht werden können. Durch Hochziehen der Indizes erhalten wir
Terme wie wir sie für unseren Beweis brauchen. Wir bezeichnen sie entsprechend mit $X^{\mu\nu}=g^{\mu\alpha}g^{\nu\beta}X_{\alpha\beta}$.
Wir extrahieren nun die Terme zweiter Ableitungen aus $G^{\mu\nu}$ und $\frac{1}{-g}\left((-g)\left(g^{\mu\nu}g^{\alpha\beta}-g^{\alpha\mu}g^{\beta\nu}\right)\right)_{,\alpha\beta}$. Da $G^{\mu\nu}=R^{\mu\nu}-\frac{1}{2}Rg^{\mu\nu}$ ist, widmen wir uns zuerst dem Ricci-Tensor $R^{\mu\nu}$:
\begin{eqnarray*}
R_{\mu\lambda} &=& -\Gamma^\nu_{\mu\nu,\lambda}+\Gamma^\nu_{\beta\lambda}\Gamma^\beta_{\mu\nu}
+\Gamma^\nu_{\mu\lambda,\nu}-\Gamma^\nu_{\beta\nu}\Gamma^\beta_{\mu\lambda}\\
&=& -\Gamma^\nu_{\mu\nu,\lambda}+\Gamma^\nu_{\mu\lambda,\nu}+...\\
&=& -\left(g^{\nu\sigma}\Gamma_{\mu\nu\sigma}\right)_{,\lambda}+\left(g^{\nu\sigma}\Gamma_{\mu\lambda\sigma}\right)_{,\nu}+...\\
&=& -g^{\nu\sigma}\Gamma_{\mu\nu\sigma,\lambda}+g^{\nu\sigma}\Gamma_{\mu\lambda\sigma,\nu}+...\\
&=& \frac{1}{2}\left(-g^{\nu\sigma}(g_{\sigma\nu,\mu\lambda}+g_{\mu\sigma,\nu\lambda}-g_{\mu\nu,\sigma\lambda}\right)
+g^{\nu\sigma}\left(g_{\sigma\lambda,\mu\nu}+g_{\mu\sigma,\lambda\nu}-g_{\mu\lambda,\sigma\nu}\right)+...\\
&=&\frac{1}{2}\left(-a^2_{\mu\lambda}-b^1_{\mu\lambda}+b^1_{\mu\lambda}+b^2_{\mu\lambda}+b^1_{\mu\lambda}-a^1_{\mu\lambda}\right)+...\\
&=&\frac{1}{2}\left(-a^1_{\mu\lambda}-a^2_{\mu\lambda}+b^1_{\mu\lambda}+b^2_{\mu\lambda}\right)+...\\
R &=& -A+B+...\\
G_{\mu\lambda} &=&\frac{1}{2}\left(-a^1_{\mu\lambda}-a^2_{\mu\lambda}+b^1_{\mu\lambda}+b^2_{\mu\lambda}+\left(A-B\right)g_{\mu\lambda}\right)+...\\
G^{\mu\lambda} &=&\frac{1}{2}\left(-a^{1 \mu\lambda}-a^{2 \mu\lambda}+b^{1 \mu\lambda}+b^{2\mu\lambda}+\left(A-B\right)g^{\mu\lambda}\right)+...
\end{eqnarray*}
Nun zu $\frac{1}{-g}\left((-g)\left(g^{\mu\nu}g^{\alpha\beta}-g^{\alpha\mu}g^{\beta\nu}\right)\right)_{,\alpha\beta}$
zunächst $\left(g^{\mu\nu}g^{\alpha\beta}-g^{\alpha\mu}g^{\beta\nu}\right)_{,\alpha\beta}$ :
\begin{eqnarray*}
\left(g^{\mu\nu}g^{\alpha\beta}-g^{\alpha\mu}g^{\beta\nu}\right)_{,\alpha\beta}&=&
g^{\mu\nu}_{,\alpha\beta}g^{\alpha\beta}+g^{\mu\nu}g^{\alpha\beta}_{,\alpha\beta}-g^{\alpha\mu}_{,\alpha\beta}g^{\beta\nu}-g^{\alpha\mu}g^{\beta\nu}_{,\alpha\beta}
\end{eqnarray*}
es gilt
\begin{eqnarray*}
g^{\mu\nu}_{,\alpha\beta} &=& -\left(g^{\mu\sigma}g^{\nu\tau}g_{\sigma\tau,\alpha}\right)_{,\beta}=-g^{\mu\sigma}g^{\nu\tau}g_{\sigma\tau,\alpha\beta}+...
\end{eqnarray*}
und somit
\begin{eqnarray*}\left(g^{\mu\nu}g^{\alpha\beta}-g^{\alpha\mu}g^{\beta\nu}\right)_{,\alpha\beta}+...
&=&g^{\mu\nu}_{,\alpha\beta}g^{\alpha\beta}+g^{\mu\nu}g^{\alpha\beta}_{,\alpha\beta}-g^{\alpha\mu}_{,\alpha\beta}g^{\beta\nu}-g^{\alpha\mu}g^{\beta\nu}_{,\alpha\beta}+...\\
&=&-g^{\mu\sigma}g^{\nu\tau}g_{\sigma\tau,\alpha\beta}g^{\alpha\beta}-g^{\mu\nu}g^{\alpha\sigma}g^{\beta\tau}g_{\sigma\tau,\alpha\beta}+...\\
&&+g^{\alpha\sigma}g^{\mu\tau}g_{\sigma\tau,\alpha\beta}g^{\beta\nu}+g^{\alpha\mu}g^{\beta\sigma}g^{\nu\tau}g_{\sigma\tau,\alpha\beta}+...\\
&=& -a^{1 \mu\nu}-Bg^{\mu\nu}+b^{1 \mu\nu}+b^{2 \mu\nu}
\end{eqnarray*}
es fehlt
\begin{eqnarray*}
(-g)_{,\alpha\beta}\left(g^{\mu\nu}g^{\alpha\beta}-g^{\alpha\mu}g^{\beta\nu}\right)
\end{eqnarray*}
für den ersten Faktor erhalten wir:
\begin{eqnarray*}
-g_{,\alpha\beta} &=& \left((-g)g^{\sigma\tau}g_{\sigma\tau,\alpha}\right)_{,\beta}=(-g)g^{\sigma\tau}g_{\sigma\tau,\alpha\beta}+...
\end{eqnarray*}
also
\begin{eqnarray*}
(-g)_{,\alpha\beta}\left(g^{\mu\nu}g^{\alpha\beta}-g^{\alpha\mu}g^{\beta\nu}\right)
&=& (-g)g^{\sigma\tau}g_{\sigma\tau,\alpha\beta}\left(g^{\mu\nu}g^{\alpha\beta}-g^{\alpha\mu}g^{\beta\nu}\right)+...\\
&=& (-g)\left(Ag^{\mu\nu}-a^{2 \mu\nu}\right)+...
\end{eqnarray*}
Wir fassen zusammen:
\begin{eqnarray*}
\left((-g)\left(g^{\mu\nu}g^{\alpha\beta}-g^{\alpha\mu}g^{\beta\nu}\right)\right)_{,\alpha\beta}&=& (-g)\left(Ag^{\mu\nu}-a^{2 \mu\nu}-a^{1 \mu\nu}-Bg^{\mu\nu}+b^{1 \mu\nu}+b^{2 \mu\nu}\right)+...\\
&=&(-g)\left(-a^{1 \mu\nu}-a^{2 \mu\nu}+b^{1 \mu\nu}+b^{2 \mu\nu}+\left(A-B\right)g^{\mu\nu}\right)+...
\end{eqnarray*}
und schließlich
\begin{eqnarray*}
t^{\mu\nu} &=& -\frac{1}{\kappa}G^{\mu\nu}+\frac{1}{2\kappa(-g)}\left((-g)\left(g^{\mu\nu}g^{\alpha\beta}-g^{\alpha\mu}g^{\beta\nu}\right)\right)_{,\alpha\beta}\\
&=&-\frac{1}{2\kappa}\left(-a^{1 \mu\lambda}-a^{2 \mu\lambda}+b^{1 \mu\lambda}+b^{2\mu\lambda}+\left(A-B\right)g^{\mu\lambda}\right)\\
&&+\frac{1}{2\kappa}\left(-a^{1 \mu\nu}-a^{2 \mu\nu}+b^{1 \mu\nu}+b^{2 \mu\nu}+\left(A-B\right)g^{\mu\nu}\right)+...\\
&=& 0+...
\end{eqnarray*}
\end{paragraph}

\begin{paragraph}{Divergenzfreiheit}
Es gilt zu zeigen, dass
\begin{eqnarray*}
((-g)(t^{\mu\nu}+T^{\mu\nu}))_{,\mu}=0
\end{eqnarray*}
Wegen der Einsteinschen Feldgleichung $G^{\mu\nu}=\kappa T^{\mu\nu}$ erhalten wir
\begin{eqnarray*}
t^{\mu\nu}+T^{\mu\nu}=\frac{1}{2\kappa(-g)}\left((-g)\left(g^{\mu\nu}g^{\alpha\beta}-g^{\alpha\mu}g^{\beta\nu}\right)\right)_{,\alpha\beta}
\end{eqnarray*}
Wir müssen also zeigen, dass
\begin{eqnarray}
\label{div_free_expr}
A^{\mu\nu} &=& (-g)(t^{\mu\nu}+T^{\mu\nu}) = \left((-g)\left(g^{\mu\nu}g^{\alpha\beta}-g^{\alpha\mu}g^{\beta\nu}\right)\right)_{,\alpha\beta}\\
A^{\mu\nu}_{,\mu} &=& \left(\left((-g)\left(g^{\mu\nu}g^{\alpha\beta}-g^{\alpha\mu}g^{\beta\nu}\right)\right)_{,\alpha\beta}\right)_{,\mu}\\
&=& \left((-g)\left(g^{\mu\nu}g^{\alpha\beta}-g^{\alpha\mu}g^{\beta\nu}\right)\right)_{,\alpha\beta\mu}=0
\end{eqnarray}
Dies lässt sich daran sehen, dass der Ausdruck in der innersten Klammer bezüglich $\beta$ und $\mu$ antisymmetrisch ist.
\end{paragraph}

\end{subsubsection}
\begin{subsubsection}{Infinitesimale Koordinatentransformation}
\begin{paragraph}{Transformations-Regeln}
Wir betrachten eine Koordinatentransformation
\begin{eqnarray*}
x^\mu &\mapsto& y^\alpha(x)
\end{eqnarray*}
und die damit verbundenen Transformationen von Tensoren, im speziellen der Metrik $g_{\mu\nu}$. Wir bezeichnen
die in den $y$-Koordinaten ausgedrückten Größen mit eine Tilde ($\tilde{g}_{\mu\nu}(y)...)$. Es muss gelten:
\begin{eqnarray*}
ds^2 &=& g_{\mu\nu}(x)dx^\mu dx^\nu=\tilde{g}_{\alpha\beta}(y(x))dy^\alpha dy^\beta
\end{eqnarray*}
wobei
\begin{eqnarray*}
dy^\alpha &=& \frac{\partial y^\alpha}{\partial x^\mu}dx^\mu
\end{eqnarray*}
also
\begin{eqnarray*}
ds^2 &=& g_{\mu\nu}(x)dx^\mu dx^\nu=\tilde{g}_{\alpha\beta}(y(x))\frac{\partial y^\alpha}{\partial x^\mu}(x)dx^\mu
\frac{\partial y^\beta}{\partial x^\nu}(x)dx^\nu
\end{eqnarray*}
Für alle $dx$. Also muss
\begin{eqnarray*}
g_{\mu\nu}(x)=\tilde{g}_{\alpha\beta}(y(x))\frac{\partial y^\alpha}{\partial x^\mu}(x)
\frac{\partial y^\beta}{\partial x^\nu}(x)
\end{eqnarray*}
gelten. Oder äquivalent dazu
\begin{eqnarray}
\label{g_transf}
\tilde{g}_{\alpha\beta}(y)=g_{\mu\nu}(x(y))\frac{\partial x^\mu}{\partial y^\alpha}(y)
\frac{\partial x^\nu}{\partial y^\beta}(y)
\end{eqnarray}
Die Transformations-Regeln für einen beliebigen Tensor $A^\mu_\nu(x)$ lauten entsprechend:
\begin{eqnarray}
\label{A_transf}
\tilde{A}^\alpha_\beta(y)=A^\mu_\nu(x(y))\frac{\partial y^\alpha}{\partial x^\mu}(x(y))
\frac{\partial x^\nu}{\partial y^\beta}(y)
\end{eqnarray}
Wir machen nun den Übergang zu einer infinitesimalen Transformation
\begin{eqnarray*}
y^\mu (x) &=& x^\mu+\xi^\mu(x)\\
x^\mu (y) &=& y^\mu-\xi^\mu(y)
\end{eqnarray*}
Entsprechend geht \ref{g_transf} in
\begin{eqnarray*}
\tilde{g}_{\alpha\beta}(y)&=&g_{\mu\nu}(x(y))\frac{\partial x^\mu}{\partial y^\alpha}(y)
\frac{\partial x^\nu}{\partial y^\beta}(y)\\
&=&g_{\mu\nu}(y-\xi)(\delta^\mu_\alpha-\xi^\mu_{,\alpha})
(\delta^\nu_\beta-\xi^\nu_{,\beta})\\
&=&\left(g_{\mu\nu}(y)-g_{\mu\nu,\sigma}\xi^\sigma(y)\right)(\delta^\mu_\alpha-\xi^\mu_{,\alpha}(y))
(\delta^\nu_\beta-\xi^\nu_{,\beta}(y))
\end{eqnarray*}
über. In erster Ordnung in $\xi$ wird daraus
\begin{eqnarray*}
\tilde{g}_{\alpha\beta}(y)&=&g_{\alpha\beta}(y)-g_{\alpha\beta,\sigma}(y)\xi^\sigma(y)
-g_{\alpha\nu}(y)\xi^\nu_{,\beta}(y)
-g_{\mu\beta}(y)\xi^\mu_{,\alpha}(y)
\end{eqnarray*}
Nun können wir $\delta g_{\alpha\beta}$ hinschreiben:
\begin{eqnarray*}
\delta g_{\alpha\beta}=\tilde{g}_{\alpha\beta}(y)-g_{\alpha\beta}(y)&=&-g_{\alpha\beta,\sigma}(y)\xi^\sigma(y)
-g_{\alpha\nu}(y)\xi^\nu_{,\beta}(y)
-g_{\mu\beta}(y)\xi^\mu_{,\alpha}(y)
\end{eqnarray*}
Die Tensor-Eigenschaft von $\delta g_{\alpha\beta}$ ist hier nicht direkt sichtbar. Wir können aber zeigen, dass
\begin{eqnarray*}
\delta g_{\alpha\beta}&=&
-g_{\alpha\nu}(y)\xi^\nu_{;\beta}(y)
-g_{\mu\beta}(y)\xi^\mu_{;\alpha}(y)
\end{eqnarray*}
gilt, woraus die Tensor-Eigenschaft direkt ersichtlich wird.
\end{paragraph}
\begin{proof}
\begin{eqnarray*}
\xi^\nu_{;\beta} &=& \xi^\nu_{,\beta}+\Gamma^\nu_{\beta\sigma}\xi^\sigma\\
\xi^\nu_{,\beta} &=& \xi^\nu_{;\beta}-\Gamma^\nu_{\beta\sigma}\xi^\sigma\\
g_{\alpha\nu}\xi^\nu_{,\beta} &=& g_{\alpha\nu}\xi^\nu_{;\beta}-\Gamma_{\beta\sigma\alpha}\xi^\sigma\\
&=& g_{\alpha\nu}\xi^\nu_{;\beta}-\frac{1}{2}\left(g_{\alpha\sigma,\beta}+g_{\beta\alpha,\sigma}-g_{\beta\sigma,\alpha}\right)\xi^\sigma\\
g_{\beta\mu}\xi^\mu_{,\alpha} &=&g_{\beta\mu}\xi^\mu_{;\alpha}-\frac{1}{2}\left(g_{\beta\sigma,\alpha}+g_{\beta\alpha,\sigma}-g_{\alpha\sigma,\beta}\right)\xi^\sigma\\
g_{\alpha\nu}\xi^\nu_{,\beta}+g_{\beta\mu}\xi^\mu_{,\alpha} &=& g_{\alpha\nu}\xi^\nu_{;\beta}+g_{\beta\mu}\xi^\mu_{;\alpha}
-g_{\beta\alpha,\sigma}\xi^\sigma\\
\delta g_{\alpha\beta}&=&-g_{\alpha\nu}\xi^\nu_{;\beta}-g_{\beta\mu}\xi^\mu_{;\alpha}
\end{eqnarray*}
\qed
\end{proof}
\end{subsubsection}


\end{subsection}
\eeeccc

\begin{appendix}
\beginchap{Ausdrücke für den Ricci-Tensor und den Ricci-Skalar}
\begin{subsection}{Der Ricci-Tensor}
Der Krümmungstensor lautet
\begin{eqnarray*}
R^\alpha_{\mu\nu\lambda} &=& -\Gamma^\alpha_{\mu\nu,\lambda}+\Gamma^\alpha_{\beta\lambda}\Gamma^\beta_{\mu\nu}
+\Gamma^\alpha_{\mu\lambda,\nu}-\Gamma^\alpha_{\beta\nu}\Gamma^\beta_{\mu\lambda}
\end{eqnarray*}
woraus sich der Ricci-Tensor ergibt
\begin{eqnarray*}
R_{\mu\lambda} = R^\nu_{\mu\nu\lambda} &=& -\Gamma^\nu_{\mu\nu,\lambda}+\Gamma^\nu_{\beta\lambda}\Gamma^\beta_{\mu\nu}
+\Gamma^\nu_{\mu\lambda,\nu}-\Gamma^\nu_{\beta\nu}\Gamma^\beta_{\mu\lambda}
\end{eqnarray*}
Dabei lässt sich
\begin{eqnarray*}
\Gamma^\nu_{\mu\nu} &=& \frac{1}{2}g^{\alpha\beta}g_{\alpha\beta,\mu}\\
\Gamma^\nu_{\mu\nu,\lambda} &=& \frac{1}{2}\left(g^{\alpha\beta}_{,\lambda}g_{\alpha\beta,\mu}+g^{\alpha\beta}g_{\alpha\beta,\mu\lambda}\right)\\
&=& \frac{1}{2}\left(-g^{\alpha\sigma}g^{\beta\tau}g_{\sigma\tau_,\lambda}g_{\alpha\beta,\mu}+g^{\alpha\beta}g_{\alpha\beta,\mu\lambda}\right)
\end{eqnarray*}
benutzen:
\begin{eqnarray*}
R_{\mu\lambda} &=& -\frac{1}{2}\left(-g^{\sigma\alpha}g^{\tau\beta}g_{\sigma\tau_,\lambda}g_{\alpha\beta,\mu}+g^{\alpha\beta}g_{\alpha\beta,\mu\lambda}\right)+\Gamma^\nu_{\beta\lambda}\Gamma^\beta_{\mu\nu}
+\Gamma^\nu_{\mu\lambda,\nu}- \frac{1}{2}g^{\alpha\gamma}g_{\alpha\gamma,\beta}\Gamma^\beta_{\mu\lambda}
\end{eqnarray*}
Die Symmetrie von $R_{\mu\lambda}$ ist nun unmittelbar ablesbar.
\end{subsection}


\begin{subsection}{Der Ricci-Skalar}
Der Ricci-Skalar entsteht durch eine weitere Kontraktion:
\begin{eqnarray*}
R = g^{\mu\lambda}R_{\mu\lambda} &=& g^{\mu\lambda}\left(-\frac{1}{2}\left(-g^{\sigma\alpha}g^{\tau\beta}g_{\sigma\tau_,\lambda}g_{\alpha\beta,\mu}+g^{\alpha\beta}g_{\alpha\beta,\mu\lambda}\right)+\Gamma^\nu_{\beta\lambda}\Gamma^\beta_{\mu\nu}
+\Gamma^\nu_{\mu\lambda,\nu}- \frac{1}{2}g^{\alpha\gamma}g_{\alpha\gamma,\beta}\Gamma^\beta_{\mu\lambda}\right)
\end{eqnarray*}
Eine genauere Betrachtung des obigen Ausdrucks zeigt, dass der Ricci-Skalar aus totalen Kontraktionen mittels $g^{\mu\nu}$ von zweiten Ableitungen $g_{\alpha\beta,\mu\nu}$ und Produkten der ersten Ableitungen $g_{\alpha\beta,\gamma}g_{\mu\nu,\tau}$ besteht. Für $g_{\alpha\beta,\mu\nu}$ gibt es nur zwei Möglichkeiten:
\begin{eqnarray*}
A &=& g^{\alpha\beta}g^{\mu\nu}g_{\alpha\beta,\mu\nu}\\
B &=& g^{\alpha\mu}g^{\beta\nu}g_{\alpha\beta,\mu\nu}
\end{eqnarray*}
Für $g_{\alpha\beta,\gamma}g_{\mu\nu,\tau}$ gibt es zwei Gruppen. Die erste besteht aus drei Kontraktionen die alle beide Faktoren überspannen:
\begin{eqnarray*}
C &=& g^{\alpha\mu}g^{\beta\nu}g^{\gamma\tau}g_{\alpha\beta,\gamma}g_{\mu\nu,\tau}\\
D &=& g^{\alpha\mu}g^{\beta\tau}g^{\gamma\nu}g_{\alpha\beta,\gamma}g_{\mu\nu,\tau}
\end{eqnarray*}
Für die zweite Gruppe wird je eine Kontraktion innerhalb der beiden Faktoren durchgeführt. Davon gibt es zwei:
\begin{eqnarray*}
\tilde{A}_\mu &=& g^{\alpha\beta}g_{\alpha\beta,\mu}\\
\tilde{B}_\mu &=& g^{\alpha\beta}g_{\alpha\mu,\beta}
\end{eqnarray*}
und wir erhalten die verbleibenden drei totale Kontraktionen:
\begin{eqnarray*}
E &=& \tilde{A}_\mu\tilde{A}_\nu g^{\mu\nu}\\
F &=& \tilde{A}_\mu\tilde{B}_\nu g^{\mu\nu}\\
G &=& \tilde{B}_\mu\tilde{B}_\nu g^{\mu\nu}
\end{eqnarray*}
Für den ersten Term in Klammer erhalten wir
\begin{eqnarray*}
g^{\mu\lambda}\left(-\frac{1}{2}\left(-g^{\sigma\alpha}g^{\tau\beta}g_{\sigma\tau_,\lambda}g_{\alpha\beta,\mu}+g^{\alpha\beta}g_{\alpha\beta,\mu\lambda}\right)\right)
&=& \frac{1}{2}\left(g^{\mu\lambda}g^{\sigma\alpha}g^{\tau\beta}g_{\sigma\tau_,\lambda}g_{\alpha\beta,\mu}-g^{\mu\lambda}g^{\alpha\beta}g_{\alpha\beta,\mu\lambda}\right)\\
&=& \frac{1}{2}(C-A)
\end{eqnarray*}
Für den zweiten:
\begin{eqnarray*}
g^{\mu\lambda}\Gamma^\nu_{\beta\lambda}\Gamma^\beta_{\mu\nu}&=& g^{\mu\lambda}g^{\nu\tau}\Gamma_{\beta\lambda\tau}g^{\beta\sigma}\Gamma_{\mu\nu\sigma}\\
&=& \frac{1}{4}g^{\mu\lambda}g^{\nu\tau}g^{\beta\sigma}\left(g_{\tau\lambda,\beta}+g_{\beta\tau,\lambda}-g_{\beta\lambda,\tau}\right)
\left(g_{\sigma\nu,\mu}+g_{\mu\sigma,\nu}-g_{\mu\nu,\sigma}\right)\\
&=& \frac{1}{4}\left(\left(D+D-C\right)+\left(C+D-D\right)-\left(D+C-D\right)\right)\\
&=&  \frac{1}{2}D- \frac{1}{4}C
\end{eqnarray*}
Für den dritten:
\begin{eqnarray*}
g^{\mu\lambda}\Gamma^\nu_{\mu\lambda,\nu} &=& g^{\mu\lambda}\left(g^{\nu\sigma}\Gamma_{\mu\lambda\sigma}\right)_{,\nu}=g^{\mu\lambda}\left(g^{\nu\sigma}_{,\nu}\Gamma_{\mu\lambda\sigma}+g^{\nu\sigma}\Gamma_{\mu\lambda\sigma,\nu}\right)\\
&=&g^{\mu\lambda}\left(g^{\nu\sigma}_{,\nu}\Gamma_{\mu\lambda\sigma}+g^{\nu\sigma}\Gamma_{\mu\lambda\sigma,\nu}\right)\\
&=&g^{\mu\lambda}\left(-g^{\nu\alpha}g^{\sigma\beta}g_{\alpha\beta,\nu}\Gamma_{\mu\lambda\sigma}+g^{\nu\sigma}\Gamma_{\mu\lambda\sigma,\nu}\right)
\end{eqnarray*}
Wir werten die beiden Terme getrennt aus:
\begin{eqnarray*}
-g^{\mu\lambda}g^{\nu\alpha}g^{\sigma\beta}g_{\alpha\beta,\nu}\Gamma_{\mu\lambda\sigma} &=& -\frac{1}{2}g^{\mu\lambda}g^{\nu\alpha}g^{\sigma\beta}g_{\alpha\beta,\nu}\left(g_{\sigma\lambda,\mu}+g_{\mu\sigma,\lambda}-g_{\mu\lambda,\sigma}\right)\\
&=&-\frac{1}{2}g^{\sigma\beta}\tilde{B}_\beta\left(2\tilde{B}_\sigma-\tilde{A}_\sigma\right)\\
&=& -G+\frac{1}{2}F\\
g^{\mu\lambda}g^{\nu\sigma}\Gamma_{\mu\lambda\sigma,\nu} &=& \frac{1}{2}g^{\mu\lambda}g^{\nu\sigma}\left(g_{\sigma\lambda,\mu\nu}+g_{\mu\sigma,\lambda\nu}-g_{\mu\lambda,\sigma\nu}\right)\\
&=& B-\frac{1}{2}A
\end{eqnarray*}
Also ergibt der dritte Term schließlich:
\begin{eqnarray*}
g^{\mu\lambda}\Gamma^\nu_{\mu\lambda,\nu} &=& -G+B+\frac{1}{2}(F-A)
\end{eqnarray*}
Und nun noch der letzte Term:
\begin{eqnarray*}
- \frac{1}{2}g^{\mu\lambda}g^{\alpha\gamma}g_{\alpha\gamma,\beta}\Gamma^\beta_{\mu\lambda} &=& - \frac{1}{2}g^{\mu\lambda}g^{\alpha\gamma}g^{\beta\sigma}g_{\alpha\gamma,\beta}\Gamma_{\mu\lambda\sigma}\\
&=& - \frac{1}{2}g^{\mu\lambda}g^{\beta\sigma}\tilde{A}_\beta\Gamma_{\mu\lambda\sigma}\\
&=& - \frac{1}{4}g^{\beta\sigma}\tilde{A}_\beta g^{\mu\lambda}\left(g_{\sigma\lambda,\mu}+g_{\mu\sigma,\lambda}-g_{\mu\lambda,\sigma}\right)\\
&=& - \frac{1}{4}g^{\beta\sigma}\tilde{A}_\beta \left(2\tilde{B}_\sigma-\tilde{A}_\sigma\right)\\
&=& - \frac{1}{2}F+\frac{1}{4}E
\end{eqnarray*}
Und somit ergibt sich $R$ schließlich als
\begin{eqnarray*}
R &=& \frac{1}{2}(C-A)+\frac{1}{2}D- \frac{1}{4}C-G+B+\frac{1}{2}(F-A) - \frac{1}{2}F+\frac{1}{4}E\\
&=& -A+B+\frac{1}{4}C+\frac{1}{2}D+\frac{1}{4}E-G
\end{eqnarray*}
\end{subsection}

\begin{subsection}{Ricci-Tensor(2)}
\label{tens2_terms}
In analoger Weise lassen sich strukturgleiche Terme des Ricci-Tensors bestimmen. Diese können aus den obigen Termen hergeleitet werden:
\begin{eqnarray*}
A &=& g^{\alpha\beta}g^{\mu\nu}g_{\alpha\beta,\mu\nu}\\
a^1_{\mu\lambda} &=& g^{\sigma\tau}g_{\mu\lambda,\sigma\tau}\\
a^2_{\mu\lambda} &=& g^{\sigma\tau}g_{\sigma\tau,\mu\lambda}\\
B &=& g^{\alpha\mu}g^{\beta\nu}g_{\alpha\beta,\mu\nu}\\
b^1_{\mu\lambda} &=& g^{\sigma\tau}g_{\mu\sigma,\lambda\tau}\\
b^2_{\mu\lambda} &=& g^{\sigma\tau}g_{\lambda\sigma,\mu\tau}=b^1_{\lambda\mu}\\
C &=& g^{\alpha\mu}g^{\beta\nu}g^{\gamma\tau}g_{\alpha\beta,\gamma}g_{\mu\nu,\tau}\\
c^1_{\mu\lambda} &=& g^{\beta\nu}g^{\gamma\tau}g_{\mu\beta,\gamma}g_{\lambda\nu,\tau}\\
c^2_{\mu\lambda} &=& g^{\beta\nu}g^{\gamma\tau}g_{\gamma\beta,\mu}g_{\tau\nu,\lambda}\\
D &=& g^{\alpha\mu}g^{\beta\tau}g^{\gamma\nu}g_{\alpha\beta,\gamma}g_{\mu\nu,\tau}\\
d^1_{\mu\lambda} &=& g^{\beta\tau}g^{\gamma\nu}g_{\mu\beta,\gamma}g_{\lambda\nu,\tau}\\
d^2_{\mu\lambda} &=& g^{\alpha\mu}g^{\gamma\nu}g_{\alpha\mu,\gamma}g_{\mu\nu,\lambda}\\
d^3_{\mu\lambda} &=& g^{\alpha\mu}g^{\gamma\nu}g_{\alpha\lambda,\gamma}g_{\mu\nu,\mu}=d^2_{\lambda\mu}\\
E &=& \tilde{A}_\mu\tilde{A}_\nu g^{\mu\nu}\\
e^1_{\mu\lambda} &=& \tilde{A}_\mu\tilde{A}_\lambda\\
e^2_{\mu\lambda} &=& g_{\mu\lambda,\sigma}g^{\sigma\tau}\tilde{A}_\tau\\
F &=& \tilde{A}_\mu\tilde{B}_\nu g^{\mu\nu}\\
f^1_{\mu\lambda} &=&  \tilde{A}_\mu\tilde{B}_\lambda\\
f^2_{\mu\lambda} &=&  \tilde{A}_\lambda\tilde{B}_\mu=f^1_{\lambda\mu}\\
f^3_{\mu\lambda} &=& g_{\mu\lambda,\sigma}g^{\sigma\tau}\tilde{B}_\tau\\
f^4_{\mu\lambda} &=& g_{\mu\sigma,\lambda}g^{\sigma\tau}\tilde{A}_\tau\\
f^5_{\mu\lambda} &=& g_{\lambda\sigma,\mu}g^{\sigma\tau}\tilde{A}_\tau=f^4_{\lambda\mu}\\
G &=& \tilde{B}_\mu\tilde{B}_\nu g^{\mu\nu}\\
g^1_{\mu\lambda} &=& \tilde{B}_\mu\tilde{B}_\lambda\\
g^2_{\mu\lambda} &=& g_{\lambda\sigma,\mu}g^{\sigma\tau}\tilde{B}_\tau\\
g^3_{\mu\lambda} &=& g_{\mu\sigma,\lambda}g^{\sigma\tau}\tilde{B}_\tau=g^2_{\lambda\mu}
\end{eqnarray*}


\end{subsection}

\eeeccc
\end{appendix}
\end{document}
